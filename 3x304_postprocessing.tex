\subsection{Post processing}

In order to achieve significant and repeatable results from measurements, care must be taken. First of all it has been discussed whether the files recorded must be filtered. Filtering is always a trade-off, in order to completely remove some frequency bands sharp filters are needed. While this is interesting for looking at different frequencies, it more or less transform our broadband algorithm into a narrow-band one. 

\subsubsection{Filter the files ?}



The min-power SRP-PHAT is looking at localizing the stationary sound sources in the far field. Indeed the cross-correlation if taken over large time windows while discard the sources that are not stationary i.e that appear as random noise in the signal. Therefore it is clear that by taking large windows, the stationary sources are going to peak higher in the cross correlation, also a higher dynamic range can be achieved. Figures \ref{fig:avoir} display the result of a real-life measurement with different file length.


\subsubsection{Effect of frequency filtering}
\subsubsection{Effect of audio recording length}
\subsubsection{Effect of high frequency content}