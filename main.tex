 \documentclass[a4paper, 12pt]{report}
%  A simple AAU report template.
%  2015-05-08 v. 1.2.0
%  Copyright 2010-2015 by Jesper Kjær Nielsen <jkn@es.aau.dk>
%
%  This is free software: you can redistribute it and/or modify
%  it under the terms of the GNU General Public License as published by
%  the Free Software Foundation, either version 3 of the License, or
%  (at your option) any later version.
%
%  This is distributed in the hope that it will be useful,
%  but WITHOUT ANY WARRANTY; without even the implied warranty of
%  MERCHANTABILITY or FITNESS FOR A PARTICULAR PURPOSE.  See the
%  GNU General Public License for more details.
%
%  You can find the GNU General Public License at <http://www.gnu.org/licenses/>.
%
%%%%%%%%%%%%%%%%%%%%%%%%%%%%%%%%%%%%%%%%%%%%%%%%
% Language, Encoding and Fonts
% http://en.wikibooks.org/wiki/LaTeX/Internationalization
%%%%%%%%%%%%%%%%%%%%%%%%%%%%%%%%%%%%%%%%%%%%%%%%
% Select encoding of your inputs. Depends on
% your operating system and its default input
% encoding. Typically, you should use
%   Linux  : utf8 (most modern Linux distributions)
%            latin1 
%   Windows: ansinew
%            latin1 (works in most cases)
%   Mac    : applemac
% Notice that you can manually change the input
% encoding of your files by selecting "save as"
% an select the desired input encoding. 
\usepackage[utf8]{inputenc}
% Make latex understand and use the typographic
% rules of the language used in the document.
\usepackage[english]{babel}
% Use the palatino font
\usepackage[sc]{mathpazo}
\usepackage{gensymb}
\linespread{1.05}         % Palatino needs more leading (space between lines)
% Choose the font encoding
\usepackage[T1]{fontenc}
%%%%%%%%%%%%%%%%%%%%%%%%%%%%%%%%%%%%%%%%%%%%%%%%
% Graphics and Tables
% http://en.wikibooks.org/wiki/LaTeX/Importing_Graphics
% http://en.wikibooks.org/wiki/LaTeX/Tables
% http://en.wikibooks.org/wiki/LaTeX/Colors
%%%%%%%%%%%%%%%%%%%%%%%%%%%%%%%%%%%%%%%%%%%%%%%%
% load a colour package
\usepackage{xcolor}
\definecolor{aaublue}{RGB}{33,26,82}% dark blue
% The standard graphics inclusion package
\usepackage{graphicx}
% Set up how figure and table captions are displayed
\usepackage{caption}
\captionsetup{%
  font=footnotesize,% set font size to footnotesize
  labelfont=bf % bold label (e.g., Figure 3.2) font
}
% Make the standard latex tables look so much better
\usepackage{array,booktabs}
% Enable the use of frames around, e.g., theorems
% The framed package is used in the example environment
\usepackage{framed}

%%%%%%%%%%%%%%%%%%%%%%%%%%%%%%%%%%%%%%%%%%%%%%%%
% Mathematics
% http://en.wikibooks.org/wiki/LaTeX/Mathematics
%%%%%%%%%%%%%%%%%%%%%%%%%%%%%%%%%%%%%%%%%%%%%%%%
% Defines new environments such as equation,
% align and split 
\usepackage{amsmath}
% Adds new math symbols
\usepackage{amssymb}
% Use theorems in your document
% The ntheorem package is also used for the example environment
% When using thmmarks, amsmath must be an option as well. Otherwise \eqref doesn't work anymore.
\usepackage[framed,amsmath,thmmarks]{ntheorem}
\usepackage{pdfpages}
%%%%%%%%%%%%%%%%%%%%%%%%%%%%%%%%%%%%%%%%%%%%%%%%
% Page Layout
% http://en.wikibooks.org/wiki/LaTeX/Page_Layout
%%%%%%%%%%%%%%%%%%%%%%%%%%%%%%%%%%%%%%%%%%%%%%%%
% Change margins, papersize, etc of the document
\usepackage[
  a4paper,
  total={170mm,250mm},
  top=28mm,
  inner=28mm,% left margin on an odd page
  outer=28mm,% right margin on an odd page
  ]{geometry}
% Modify how \chapter, \section, etc. look
% The titlesec package is very configureable
\usepackage{titlesec}
\titleformat{\chapter}[display]{\normalfont\huge\bfseries}{\chaptertitlename\ \thechapter}{20pt}{\Huge}
\titleformat*{\section}{\normalfont\Large\bfseries}
\titleformat*{\subsection}{\normalfont\large\bfseries}
\titleformat*{\subsubsection}{\normalfont\normalsize\bfseries}
%\titleformat*{\paragraph}{\normalfont\normalsize\bfseries}
%\titleformat*{\subparagraph}{\normalfont\normalsize\bfseries}

% Clear empty pages between chapters
\let\origdoublepage\cleardoublepage
\newcommand{\clearemptydoublepage}{%
  \clearpage
  {\pagestyle{empty}\origdoublepage}%
}
\let\cleardoublepage\clearemptydoublepage

% Change the headers and footers
\usepackage{fancyhdr}
\pagestyle{fancy}
\fancyhf{} %delete everything
\renewcommand{\headrulewidth}{0pt} %remove the horizontal line in the header
\fancyhead[RE]{\small\nouppercase\leftmark} %even page - chapter title
\fancyhead[LO]{\small\nouppercase\rightmark} %uneven page - section title
\fancyhead[LE,RO]{\thepage} %page number on all pages
% Do not stretch the content of a page. Instead,
% insert white space at the bottom of the page
\raggedbottom
% Enable arithmetics with length. Useful when
% typesetting the layout.
\usepackage{calc}


%%%%%%%%%%%%%%%%%%%%%%%%%%%%%%%%%%%%%%%%%%%%%%%%
% Misc
%%%%%%%%%%%%%%%%%%%%%%%%%%%%%%%%%%%%%%%%%%%%%%%%
% Add bibliography and index to the table of
% contents
\usepackage[nottoc]{tocbibind}
% Add the command \pageref{LastPage} which refers to the
% page number of the last page
\usepackage{lastpage}
% Add todo notes in the margin of the document
\usepackage[
%  disable, %turn off todonotes
  colorinlistoftodos, %enable a coloured square in the list of todos
  textwidth=\marginparwidth, %set the width of the todonotes
  textsize=scriptsize, %size of the text in the todonotes
  ]{todonotes}

%%%%%%%%%%%%%%%%%%%%%%%%%%%%%%%%%%%%%%%%%%%%%%%%
% Hyperlinks
% http://en.wikibooks.org/wiki/LaTeX/Hyperlinks
%%%%%%%%%%%%%%%%%%%%%%%%%%%%%%%%%%%%%%%%%%%%%%%%
% Enable hyperlinks and insert info into the pdf
% file. Hypperref should be loaded as one of the 
% last packages
\usepackage{hyperref}
\hypersetup{%
	pdfpagelabels=true,%
	plainpages=false,%
	pdfauthor={Author(s)},%
	pdftitle={Title},%
	pdfsubject={Subject},%
	bookmarksnumbered=true,%
	colorlinks=false,%
	citecolor=black,%
	filecolor=black,%
	linkcolor=black,% you should probably change this to black before printing
	urlcolor=black,%
	pdfstartview=FitH%
}


\usepackage{amsmath}
\usepackage{amssymb}
%  A simple AAU report template.
%  2015-05-08 v. 1.2.0
%  Copyright 2010-2015 by Jesper Kjær Nielsen <jkn@es.aau.dk>
%
%  This is free software: you can redistribute it and/or modify
%  it under the terms of the GNU General Public License as published by
%  the Free Software Foundation, either version 3 of the License, or
%  (at your option) any later version.
%
%  This is distributed in the hope that it will be useful,
%  but WITHOUT ANY WARRANTY; without even the implied warranty of
%  MERCHANTABILITY or FITNESS FOR A PARTICULAR PURPOSE.  See the
%  GNU General Public License for more details.
%
%  You can find the GNU General Public License at <http://www.gnu.org/licenses/>.
%
%
%
% see, e.g., http://en.wikibooks.org/wiki/LaTeX/Customizing_LaTeX#New_commands
% for more information on how to create macros

%%%%%%%%%%%%%%%%%%%%%%%%%%%%%%%%%%%%%%%%%%%%%%%%
% Macros for the titlepage
%%%%%%%%%%%%%%%%%%%%%%%%%%%%%%%%%%%%%%%%%%%%%%%%
%Creates the aau titlepage
\newcommand{\aautitlepage}[3]{%
  {
    %set up various length
    \ifx\titlepageleftcolumnwidth\undefined
      \newlength{\titlepageleftcolumnwidth}
      \newlength{\titlepagerightcolumnwidth}
    \fi
    \setlength{\titlepageleftcolumnwidth}{0.5\textwidth-\tabcolsep}
    \setlength{\titlepagerightcolumnwidth}{\textwidth-2\tabcolsep-\titlepageleftcolumnwidth}
    %create title page
    \thispagestyle{empty}
    \noindent%
    \begin{tabular}{@{}ll@{}}
      \parbox{\titlepageleftcolumnwidth}{
        \iflanguage{danish}{%
          \includegraphics[width=\titlepageleftcolumnwidth]{figures/aau_logo_da}
        }{%
          \includegraphics[width=\titlepageleftcolumnwidth]{aau_logo_en}
        }
      } &
      \parbox{\titlepagerightcolumnwidth}{\raggedleft\sf\small
        #2
      }\bigskip\\
       #1 &
      \parbox[t]{\titlepagerightcolumnwidth}{%
      \textbf{Abstract:}\bigskip\par
        \fbox{\parbox{\titlepagerightcolumnwidth-2\fboxsep-2\fboxrule}{%
          #3
        }}
      }\\
    \end{tabular}
    \vfill
    \iflanguage{danish}{%
      \noindent{\footnotesize\emph{Rapportens indhold er frit tilgængeligt, men offentliggørelse (med kildeangivelse) må kun ske efter aftale med forfatterne.}}
    }{%
      \noindent{\footnotesize\emph{The content of this report is freely available, but publication (with reference) may only be pursued due to agreement with the author.}}
    }
    \clearpage
  }
}

%Create english project info
\newcommand{\englishprojectinfo}[8]{%
  \parbox[t]{\titlepageleftcolumnwidth}{
    \textbf{Title:}\\ #1\bigskip\par
    \textbf{Theme:}\\ #2\bigskip\par
    \textbf{Project Period:}\\ #3\bigskip\par
    \textbf{Project Group:}\\ #4\bigskip\par
    \textbf{Participant(s):}\\ #5\bigskip\par
    \textbf{Supervisor(s):}\\ #6\bigskip\par
    \textbf{Copies:} #7\bigskip\par
    \textbf{Page Numbers:} \pageref{LastPage}\bigskip\par
    \textbf{Date of Completion:}\\ #8
  }
}

%Create danish project info
\newcommand{\danishprojectinfo}[8]{%
  \parbox[t]{\titlepageleftcolumnwidth}{
    \textbf{Titel:}\\ #1\bigskip\par
    \textbf{Tema:}\\ #2\bigskip\par
    \textbf{Projektperiode:}\\ #3\bigskip\par
    \textbf{Projektgruppe:}\\ #4\bigskip\par
    \textbf{Deltager(e):}\\ #5\bigskip\par
    \textbf{Vejleder(e):}\\ #6\bigskip\par
    \textbf{Oplagstal:} #7\bigskip\par
    \textbf{Sidetal:} \pageref{LastPage}\bigskip\par
    \textbf{Afleveringsdato:}\\ #8
  }
}

\makeatletter
\newsavebox\myboxA
\newsavebox\myboxB
\newlength\mylenA

\newcommand*\xoverline[2][0.75]{%
    \sbox{\myboxA}{$\m@th#2$}%
    \setbox\myboxB\null% Phantom box
    \ht\myboxB=\ht\myboxA%
    \dp\myboxB=\dp\myboxA%
    \wd\myboxB=#1\wd\myboxA% Scale phantom
    \sbox\myboxB{$\m@th\overline{\copy\myboxB}$}%  Overlined phantom
    \setlength\mylenA{\the\wd\myboxA}%   calc width diff
    \addtolength\mylenA{-\the\wd\myboxB}%
    \ifdim\wd\myboxB<\wd\myboxA%
       \rlap{\hskip 0.5\mylenA\usebox\myboxB}{\usebox\myboxA}%
    \else
        \hskip -0.5\mylenA\rlap{\usebox\myboxA}{\hskip 0.5\mylenA\usebox\myboxB}%
    \fi}
\makeatother

%%%%%%%%%%%%%%%%%%%%%%%%%%%%%%%%%%%%%%%%%%%%%%%%
% An example environment
%%%%%%%%%%%%%%%%%%%%%%%%%%%%%%%%%%%%%%%%%%%%%%%%
\theoremheaderfont{\normalfont\bfseries}
\theorembodyfont{\normalfont}
\theoremstyle{break}
\def\theoremframecommand{{\color{gray!50}\vrule width 5pt \hspace{5pt}}}
\newshadedtheorem{exa}{Example}[chapter]
\newenvironment{example}[1]{%
		\begin{exa}[#1]
}{%
		\end{exa}
}

\usepackage[utf8]{inputenc}
\usepackage{array}
\usepackage{siunitx} % adds SI units
\usepackage{placeins} % includes FloatBarrier
\usepackage{graphicx}
\usepackage{epstopdf}
\usepackage{caption}
\usepackage{subcaption}
\usepackage{amsmath}
\usepackage{amsfonts}
\usepackage{mathtools}
\usepackage{float}
\usepackage{gensymb}
\usepackage{csvsimple}
\usepackage{hhline}
\usepackage{pdfpages}
\usepackage[utf8]{inputenc} 
\usepackage[T1]{fontenc}
\usepackage{stix} 
\usepackage{gensymb}
\usepackage{listings}
\usepackage{color} %red, green, blue, yellow, cyan, magenta, black, white
\definecolor{mygreen}{RGB}{28,172,0} % color values Red, Green, Blue
\definecolor{mylilas}{RGB}{170,55,241}

\usepackage[backend=biber, sorting=none]{biblatex}
\usepackage{booktabs}
\addbibresource{articles.bib}
\renewcommand{\thesection}{\hspace*{-1.0em}}
\renewcommand{\thesection}{\arabic{section}}

\newcommand{\argmin}{\operatornamewithlimits{argmin}}
\newcommand{\argmax}{\operatornamewithlimits{argmax}}

\lstset{language=Matlab,%
    %basicstyle=\color{red},
    breaklines=true,%
    morekeywords={matlab2tikz},
    keywordstyle=\color{blue},%
    morekeywords=[2]{1}, keywordstyle=[2]{\color{black}},
    identifierstyle=\color{black},%
    stringstyle=\color{mylilas},
    commentstyle=\color{mygreen},%
    showstringspaces=false,%without this there will be a symbol in the places where there is a space
    numbers=left,%
    numberstyle={\tiny \color{black}},% size of the numbers
    numbersep=9pt, % this defines how far the numbers are from the text
    emph=[1]{for,end,break},emphstyle=[1]\color{red}, %some words to emphasise
    %emph=[2]{word1,word2}, emphstyle=[2]{style},    
}

\usepackage{etoolbox}
\usepackage{tocloft}


%\begin{titlepage}
%	\rule{1pt}{1.1\textheight} % Vertical line
%	\hspace{0.02\textwidth} 
%	\parbox[b]{0.95\textwidth}{ \\
%		
%		{\huge \bfseries Report\\[\baselineskip]}
%		{\large SDM vs Binaural:\\ A comparison of different room auralization techniques}\\[2\baselineskip]
%		{\large\textit{Acoustics and Audio Technology }\\{9th semester project}}\\
%		{\large Aalborg University\\[4\baselineskip]}
%		
%		\vspace{0.6\textheight}
%		{\noindent Ashwin~~~~~~~~~~~~~~~~~~~~~~~~~~~~~~~~~~~~~~~~~~ Maxime}\\[0\baselineskip]
%		{\noindent Saraf~~~~~~~~~~~~~~~~~~~~~~~~~~~~~~~~~~~~~~~~~~~ Démurger}}
%
%\end{titlepage} 







\begin{document}
\section*{Preface}\\

\vspace{1cm}

This report has been carried out during Spring of 2018 as a Acoustics and Audio Technology Master's Thesis at Aalborg University by group 10GR1062.\\

\noindent
The group would like to thank Søren Krarup Olesen (Associate Professor, AAU) and Karim Haddad (Research engineer, Brüel \& Kjær)  for their supervision throughout the project.\\

\noindent
The figures in the report are produced by the group unless a source is specified.\\

\noindent

\newcommand{\doubleSignature}[5]{
\begin{minipage}[c]{\textwidth}
\vspace{2cm}

\makebox[12cm][c]{
 #1, \today 
}
\vspace{3cm}

\makebox[12cm][c]{
\hfill \makebox[5cm][c] {\hrulefill} \hfill \makebox[5cm][c] {\hrulefill} \hfill
}
\makebox[12cm][c]{
\hfill #2 \hfill #3 \hfill
}
\makebox[12cm][c]{
\hfill #4 \hfill #5 \hfill
}
\vspace{1cm}
\end{minipage}
}

\doubleSignature{Aalborg}{Ashwin Saraf}{Maxime Démurger}{asaraf16@student.aau.dk}{mdemur16@student.aau.dk}


\tableofcontents
\chapter{Introduction}

\section{Motivation}
Sound has been a subject of fascination for a long time and with good reason. We have known since prehistoric times that sound travels slower than light, evermore proven whenever a flash of lightning was seen before the clap was heard \cite{ampel1993history}.  In 1636AD, Marin Mersènne of Paris used a pendulum to make speed of sound measurements by firing cannons. The $\rom{19}^{th}$ century mark the first attempts to create transducers and Graham Bell invention of the phone in 1876 mark a clear step forward in the technology allowing to transmit intelligible sounds. While the transducer technology was slowly developing, the inability to process the data limited the development of sound localization tool. The beginnings of sound localization can be trace back during the $\rom{20}^{th}$ century war, when rudimentary systems were developed to localize incoming enemy airplanes, giant rotating waveguides were used by an operator to steer and amplify the sound arriving from a given direction, the prequel of beamforming. For a long time, ears were the only tool to localize sound, until the development of computer and array signal processing techniques. Sound localization technology has matured since then and is now part of our daily life and implemented in countless products such as hearing aid, headset, etc.. making our life much easier in the end. This technology can now be used to solve new problems such as accessing the impact of environmental noise on human. This impact is still been studied by researchers around the world \footnote{some study claim that sound can cause health issue. Regulation are still being written to quantify a safe daily noise exposure} and a reliable noise monitoring system is yet to be developed to detect the main noise contributors in an outdoor environment. This thesis tackle the problem of localizing and quantifying main noise disturbance in outdoor environment. The main challenges of sound localization arise in developing a robust technique to localize multiple sources in a changing complex outdoor sound field. This thesis propose a method to solve this problem.

\section{Background}

Sound localization algorithms have been successfully applied (to varying degrees) to a wide range of engineering problems. Traditionally, distinction is usually made between algorithm using the time difference of arrival (TDOA) \nomenclature{\textbf{TDOA}}{Time Difference Of Arrival}  of signal between pairs microphones to find the position of a source and the algorithm using beamforming Steered Response Power (SRP). However, the SRP-PHAT algorithm, one of the most robust and widely implemented technique combines the advantages of those two techniques. A significant bit of research has been done on implementing SRP-PHAT on speech enhancement systems, whereby speaker identification and teleconferencing in an environment having high background noise and reverberant conditions was needed. Outdoor sound can be appreciably different from this situation. This thesis proposes a method to adapt the SRP-PHAT to compute a sound map of a multi-source outdoor environment.

\begin{figure}[H]
    \centering
    \includegraphics[width=0.8\textwidth]{Figures/scenariofarfield.png}
    \caption{Various outdoor sound sources being localized by a microphone array}
    \label{fig:Introductioncase}
\end{figure}

\section{Scope and outline of the thesis}

This thesis focus mainly on source signal spectrum ranging from low-to-mid frequencies in the far field. It should be noted that the purpose of this thesis is not to track moving sources, rather the thesis tackles the problem of \textit{static outdoor sound levels and their contributors} with the constraint of using as few microphones as possible while being robust to different noise and weather conditions. The thesis propose a solution capable of retrieving the noise source position in a variety of scenario including free-field to moderate reverberating environments. While previous research mostly tackle the problem of single source localization using linear or circular arrays, this thesis use a tetrahedral array to capture the signals.

The organization of the thesis is as follows: Chapter 2 derives the theory used to analyze the problem and create our simulation framework. Chapter 3 focus on describing the methods employed to solve the problem as well as algorithm features and its robustness and performance in outdoor conditions the algorithm performance.
Chapter 4 contains experimental results, anechoic and outdoor measurements investigating the algorithm limits in a variety of scenario.
Chapter 5 encompass a discussion about the solution and propose new ideas and further work.





\subsection{Direction of arrival using time delay information}

\subsubsection{TDOA of a single microphone pair}\label{sec:TDOA}

When using a pair of microphones, sound from a particular source arrives at the two microphones at different times, based on the source distance to the particular microphone. For a pair of microphones located at $m_{1}$ and $m_{2}$, the time difference of arrival (TDOA) of a sound signal from a source located at s can be defined as:
\begin{equation}
    \begin{split}
    T(\{m_{1},m_{2}\},s)&=\frac{|s-m_{1}|-|s-m_{2}|}{c}\\
                        &=\frac{|D_{1}|-|D_{2}|}{c}
    \label{eq:tdoa}
    \end{split}
\end{equation}
where c is the speed of sound in the medium and $|D_{1}|$ and $|D_{2}|$ the distance between the source and the microphones at $m_1$ and $m_2$.

In 2D, this equation leads to a hyperbola (Fig.\ref{eq:tdoa}) where the two focus points of the hyperbola are the sensors. The difference of the distance from any point of the hyperbola to the two focus is always the same.

\begin{figure}[H]
    \centering
    \includegraphics[width=0.8\textwidth]{Figures/hyperbola.png}
    \caption{A hyperbola (represented in blue), the red dot is any point on the hyperbola, the black dots represent the two foci. For any point on the hyperbola, $|D_1|-|D_2| = constant$}
    \label{eq:tdoa}
\end{figure}

In 3D, the TDOA information can be used to locate the source on a two-sheeted hyperboloid $\chi(\{m_{1},m_{2}\},s)$ such that the microphone positions are its foci. In practice the two-sheeted hyperboloid can be approximated to a cone so as to have a much simpler equation for the locus: $\theta$  = \textit{constant}, where $\theta$ is the angle of the source to the  midpoint of the line segment joining the two microphones (Fig. \ref{fig:hyperboloid_Cone}).

\begin{figure}[H]
    \centering
    \includegraphics[width=0.8\textwidth]{Figures/hyperboloid.png}
    \caption{A 2-sheeted hyperboloid with a cone approximation overlay. As the tips of the hyperbola get closer (the microphones are closer), the hyperbola approximates the cone better.}
    \label{fig:hyperboloid_Cone}
\end{figure}

Of course as the source location gets closer to being orthogonal to the midpoint of the line segment joining the two microphones ($\theta=90\degree$), the hyperbola gets wider and flatter (more planar) and approximates the cone better. Also as the source gets closer to the line joining the two microphones ($\theta=0\degree, 180\degree$), the hyperbola collapses to a straight line and approximates the cone better. Thus, the error minimizes for broad-side sound source ($\theta=90\degree$) and for end-side sound source ($\theta=0\degree, 180\degree$), and maximizes for the midsection ($\theta=45\degree, 135\degree$). The equation for the error is given by
\begin{equation}
\begin{split}
    max\{\theta_{error}\} \approx  \frac{M_{dist}^2}{16R^2}\\
    max\{D_{error}\} \approx  \frac{M_{dist}^2}{16R},
\end{split}
\end{equation}
where $D_{error}$ is the actual source distance error (the gap between the cone and the hyperboloid) \cite{Brandstein:1995:FSS:922154}.

Microphones in microphone arrays are usually closely spaced with respect to the actual source distance, so the cone approximation works well. In most scenarios errors due to noise from other system parameters are greater than the errors associated with this approximation. Thus, given the time delay information between a microphone pair, the source can be located at a particular direction $\theta$, associated with the cone for that time delay, \textit{d}. 

It can be shown that the cone approximation is the same as a far-field assumption for the sound source. The far-field assumption leads to a planar wave-front for a uniform linear microphone array. Thus, for the far-field, the DOA estimation problem is essentially the same as the TDOA estimation problem as there is a one-to-one relation between $\theta$ and \textit{d}.

\begin{figure}
    \centering
    \includegraphics[width=0.6\textwidth]{Figures/Far-field.png}
    \caption{With far-field approximation it can be assumed that the sound waves incident on the pair of microphones are parallel (planar incidence).}
    \label{fig:my_label}
\end{figure}

Now, given the TDOA between multiple microphone pairs, the source localization problem can be solved by triangulation. This triangulation problem can be solved for different variables ($\theta$, Source Distance or the Time Delay itself). The next section will detail the basics of solving such a problem. 

\subsubsection{TDOA of a tetrahedral array}\label{sec:TDOAN}

A tetrahedron simplest spatial structure with 4 vertices (a triangular pyramid). The tetrahedron has 4 vertices, 4 faces and 6 edges. If the tetrahedron is regular then all the vertices are equally spaced from each other and every face is an equilateral triangle. A 1m aperture tetrahedral microphone array composed of 4 microphones at position $p_1, p_2, p_3, p_4$ is described by fig. \ref{fig:regulartetra}

%%\begin{equation}
%%    p_1=\begin{bmatrix}-0.5 \\ 0 \\ 0\end{bmatrix}   
%%    \hspace{0.8cm}
%%    p_2=\begin{bmatrix}0.5 \\ 0 \\ 0\end{bmatrix}
%%    \hspace{0.8cm}
%%    p_3=\begin{bmatrix}0\\ -0.87 \\ 0\end{bmatrix}
%%    \hspace{0.8cm}
%%    p_4=\begin{bmatrix}0 \\ -0.43\\0.70\end{bmatrix}
%%\end{equation} 
%%%
\begin{figure}[H]
    \centering
    \begin{subfigure}[b]{0.48\textwidth}
    \centering
    \includegraphics[width=1\textwidth]{Figures/unit_tetra.png}
    \caption{Side view}
    \label{fig:d1}
\end{subfigure}
\hfill
\begin{subfigure}[b]{0.48\textwidth}
    \centering
    \includegraphics[width=1\textwidth]{Figures/unit_tetra_top2.png}
    \caption{Top view}
    \label{fig:d2}
\end{subfigure}
    \caption{Representation of a regular tetrahedron with unit sides, centroid at origin and horizontally level lower face, front view and top view}
    \label{fig:regulartetra}
\end{figure}

Let's define three Cartesian axis X,Y,Z with spherical coordinates as shown in figure \ref{fig:3daxis}.

\begin{figure}[H]
    \centering
    \includegraphics[width=0.6\textwidth]{Figures/3dCoord.png}
    \caption{Cartesian axis used throughout the thesis}
    \label{fig:3daxis}
\end{figure}

A unit vector $\hat{u}$ in this 3d space can be defined in spherical coordinates by (1,$\theta$,$\phi$), where the magnitude of the vector is 1, $\theta$ the azimuth and $\phi$ the elevation as described in figure \ref{fig:3daxis}.
\begin{equation}
    \begin{split}
        x_u&=cos(\theta)cos(\phi) \\
        y_u&=sin(\theta)cos(\phi) \\
        z_u&=sin(\phi),
    \end{split}
\end{equation}
this can be denoted by the unit propagation vector ${\hat{a}(\theta,\phi)}$ in the Cartesian coordinates
\begin{equation}
    \hat{a}(\theta,\phi)=\begin{bmatrix}cos(\theta)cos(\phi) \\sin(\theta)cos(\phi) \\sin(\phi)\end{bmatrix},
\end{equation}
Suppose sound is travelling along the unit vector $\hat{u}$. Let 2 microphones be placed at positions $\hat{p}_1=(x_1,y_1,z_1)$ and $\hat{p}_2=(x_2,y_2,z_2)$.  Then $\hat{p}_{12}=\hat{p}_{2}-\hat{p}_{1}$, where ${\hat{p_{12}}}$ is the distance between the two microphones. The projection of this distance in the direction of $\hat{u}$ is simply $\hat{a}(\theta,\phi)\cdot\hat{p}_{12}$ and the time it takes for sound to travel between the two microphones is then 
\begin{equation}
    \hat{t}_{12}=\frac{\hat{a}(\theta,\phi)\cdot\hat{p}_{12}}{c},
\end{equation} c being the speed of sound. In case of M microphones in arbitrary 3D position, $^MC_2$ combinations of pairs are possible, giving $^MC_2$ possible TDOA estimates. However, not all the pairs are linearly independent. Given 3 microphones i, j and k, the time delays $\hat{t}$ satisfy:
\begin{equation}
    \hat{t}_{i,j}=\hat{t}_{i,k}+\hat{t}_{j,k}
    \label{Eq:linearDep}
\end{equation}
Thus, only M-1 linearly independent combinations of microphone pairs exists. This can be written as the vector $\hat{t}=(\hat{t}_{1,2}.....\hat{t}_{1,M})^T$ in the M-1 dimensional subspace S $\subset$ $^MC_2$. Thus, given a tetrahedral array, only 3 linearly independent combinations of cross-correlations exist. A least-Squares approach could be considered, with the 3 use-able combinations. The `correct' DOA is then the $a(\Theta,\Phi)$ which minimizes the cost function
\begin{equation}
    J(\Theta,\Phi) = \sum\limits_{j=2}^4\bigg(T_{1j}-\frac{\hat{a}(\Theta,\Phi)\hat{p}_{1j}}{c}\bigg)^2.
    \label{Eq:linearDepSol}
\end{equation}


\iffalse
Each pair i of sensor gives a locus $\chi{i}(\{m_{i1},m_{i2}\},s)$ on which the source can be located. When more than one pair of sensor is used, the position of the source can be found at the intersection of each locus. Depending on the position of each sensor pair and the direction of arrival of the source, the intersection of all the locus could in theory be found $ ( s\in {\bigcap}_{i=0}^k \chi{i} )$. In practical, the DOA is corrupted by random noise process which influence the localization precision and therefore the locus of each microphone pair. In most case, the locus intersection is the empty set. $ ( {\bigcap}_{i=0}^k \chi{i} = \emptyset ) $


Figure \ref{fig:hyperboloid_intersect} gives a 2D representation of 3 hyperboloid intersecting. The problem is therefore how to estimate the best intersection and how does this estimation influence our TDOA. Solution to this problem are discussed in section \ref{sec:LSTDOA}

\begin{figure}[H]
    \centering
    \includegraphics[width=0.8\textwidth]{Figures/intersect.png}
    \caption{2D representation of 3 locus. As seen above the intersection of the 3 locus is the empty set. $M_{1}$ , $M_{2}$ and $M_{3}$ represent the sensor position}
    \label{fig:hyperboloid_intersect}
\end{figure}

\subsection{Least Square problem}\label{sec:LSTDOA}

Let's define a Least Square (LS) problem which solves the localization of the sources in space as explained in section \ref{sec:TDOAN}. The LS problem optimize the position of the source in space by minimizing a given error criterion $ J $.
\begin{equation}
\hat{s}= \argmin_{s} J(s) 
\end{equation}

3 error criteria can be selected for solving the source location LS problem:  $J_{TDOA}(s)$, $J_{DOA}(s)$, $J_{D}(s)$. Those criteria are explained in the following sections.

\subsubsection{$J_{TDOA}(s)$ Error Criterion}

$J_{TDOA}(s)$ is the squared error difference between the time delay estimate $\tau$ and the time delay measured between the microphone pairs. The criterion and the optimization problem is given in the following.

\begin{equation}
J_{TDOA}(s) = {\sum}_{i=0}^k \epsilon_{itdoa}.[\tau_{i}-T(\{m_{i1},m_{i2}\},s)]^2
\label{eq:jtdoa}
\end{equation}
\begin{equation}
\hat{s}_{tdoa}= \argmin_{s} J_{tdoa}(s) 
\end{equation}

Assuming that the TDOA estimates $\tau_{i}$ are independently corrupted by zero-mean white gaussian noise, the stochastic variable $\mathcal{T}_{i}$ associated with this random process follows a normal distribution. The likelihood function of such a distribution is well-known and therefore the log of the distribution likelihood can be maximized yielding the Maximum Likelihood (ML) estimate of the TDOA from which the estimated position can be computed . 

Note that the error criterion is scaled by a factor $\epsilon_{itdoa}$ which is the inverse of the TDOA estimate $\mathcal{T}_{i}$ variance. 

\begin{equation}
\epsilon_{itdoa}=\frac{1}{\mathrm{Var}{\mathcal{T}_{i}}}
\label{eq:epsilonjtdoa}
\end{equation}

\subsubsection{$J_{DOA}(s)$ Error Criterion}

This is a classical formulation of the problem, which follows the same idea as the TDOA error criterion but this time the $J_{DOA}(s)$ is the squared error difference between the DOA estimate $\theta$ and the DOA measured between the microphone pairs. 

\begin{equation}
J_{DOA}(s) = {\sum}_{i=0}^k \epsilon_{idoa}.[\Theta_{i}-\theta(\{m_{i1},m_{i2}\},s)]^2
\label{eq:jdoa}
\end{equation}
\begin{equation}
\hat{s}_{doa}= \argmin_{s} J_{doa}(s) 
\end{equation}


Assuming that the DOA estimates $\theta_{i}$ are independently corrupted by zero-mean additive white Gaussian noise, the stochastic variable $\Theta_{i}$ associated with this random process follow a normal distribution. The error criterion is also scaled by a factor $\epsilon_{idoa}$ which is the inverse of the DOA estimate $\Theta_{i}$ variance. 

\begin{equation}
\epsilon_{idoa}=\frac{1}{\mathrm{Var}{\Theta_{i}}}
\label{eq:epsilonjtdoa}
\end{equation}

\subsubsection{$J_{D}(s)$ Error Criterion}

finally, $J_{D}(s)$ is the squared error difference between the orthogonal distance from s to the appropriate cone approximation 

\begin{equation}
J_{DOA}(s) = {\sum}_{i=0}^k \epsilon_{id}.[D(\chi_{i},s)]^2
\label{eq:jd}
\end{equation}
\begin{equation}
\hat{s}_{d}= \argmin_{s} J_{d}(s) 
\end{equation}


\begin{equation}
D(\chi_{i},s)=R_{i}.\sin{[\theta_{i}-\Theta(m_{i1},m_{i2},s)}]  
\end{equation}

\begin{equation}
 \epsilon_{id}=\frac{1}{\mathrm{Var}{\Theta_{i}}}    
\end{equation}

\subsubsection{Estimator performance}

The estimators described above vary in term of performance under certain conditions. For a bi-linear array, Brandstein has shown that $J_{TDOA}(s)$ and $J_{DOA}(s)$ are more robust to great angle of incidence than $J_{D}(s)$. At low noise level, $J_{TDOA}(s)$ is proven to be slightly better than $J_{DOA}(s)$ but $J_{DOA}(s)$ perform better overall especially for long range source location. $J_{TDOA}(s)$ is better for broadside sources and low noise level but $J_{DOA}(s)$ is more robust for less favorable noise conditions

%Pros and Cons of each estimators are sumarized in the following table.
\fi
\newpage
\subsection{Outdoor sound field modelling}
Various different models have been designed for outdoor sound field received at a receiver. The ISO 9613-2 \cite{ISO9613} is an international standard model for attenuation of sound when propagating outdoors. The standard uses an empirical method to quantify attenuation in different circumstances. This is a disadvantage as the model might not fit particular real world scenarios and user discretion is needed when using the model. NMPB-2008 \cite{dutilleux2010nmpb} is a French standard model which uses simple engineering methods to model road traffic noise. Over time it has been extended to include other sound sources. Nord2000 \cite{plovsing2000nord2000} and Harmonoise \cite{defrance2007outdoor} are more advanced engineering models for outdoor sound propagation. Nord2000 was developed in the period 1996-2001 by DELTA (Denmark, project manager, SINTEF (Norway), and SP (Sweden). Harmonoise is a more recent method and is made with a collaboration of various European countries. Nord2000 and Harmonoise are based on a similar approach and often produce quite similar models. Various inconclusive studies have been conducted comparing the two \cite{garg2014critical},\cite{jonsson2008comparison}. Eventually, to have a harmonized and coherent approach, a common framework for noise assessment (CNOSSOS-EU) was developed by the European Commission \cite{kephalopoulos2012common} in co-operation with the EU Member States to be applied for strategic noise mapping as required by the Environment Noise Directive (2002/49/EC). CNOSSUS-EU investigates the various existing methods and their advantages and disadvantages. It takes into consideration the accuracy as well as the computational complexities of the various methods. In general, the effect of different factors on outdoor sound propagation are described below.
\subsubsection{Spreading loss} 
The sound intensity from an omni-directional sound source drops as a function of distance due to wavefront spreading. The intensity I at distance r of a source with power P, is given by
\begin{equation}
    I = \frac{P}{4\pi r^2}.
\end{equation}
This is due to spherical propagation, where the surface of the sphere has area $4\pi r^2$. In logarithm form this becomes
\begin{equation}
\begin{split}
    10log(I) &= 10log(\frac{P}{4\pi r^2}) \\
    L_p &= L_w - 20 log(r) - 11,
\end{split}
\end{equation}
which means a reduction of $20log2 = 6 dB$, every doubling of r. This equation assumes uniform omni-directional directivity. For directional sources a Directivity Index DI can be added giving
\begin{equation}
    L_p = L_w + DI - 20 log(r) - 11.
\end{equation}
It is important to remember that such a directivity can be inherent to the source or might be induced due to the location of the source. An omni-directional source placed on a perfectly reflecting plane can only propagate sound into a hemisphere, in which case the DI is 3 dB.  
An infinite line source can be viewed as a linear array of omni-directional point sources. The wavefront spread is cylindrical (surface area $= 2\pi r$),  which gives
\begin{equation}
\begin{split}
    10log(I) &= 10log(\frac{P}{2\pi r}) \\
    L_p &= L_w - 10 log(r) - 8,
\end{split}
\end{equation}
The DI is again 3 dB and the reduction is $10log2 = 3$ dB, every doubling of r. Highway traffic is modelled in a similar manner, assuming 3 dB drop every doubling of distance.
\begin{figure}
\includegraphics[width=0.8\textwidth]{Figures/airAbsorption.png}
\caption{Total absorption of sound in air as a function of frequency. The curves range from 0 to 100\% relative humidity and are for 20\degree C \cite{evans1972atmospheric} (Notice that the y-axis units are per 1000ft).}
\label{Fig:airAbsorption}
\end{figure}
\subsubsection{Atmospheric absorption}
Sound energy converts to heat as it travels through air. The conversion of sound-to-heat in air can happen due to conduction, shear viscosity or by molecular relaxation. The portion of sound absorbed by air becomes increasingly important as distance of propagation increases. For a plane wave, the loudness $L$ at a distance $x$ from a position of known loudness $L_0$ is given by
\begin{equation}
    L= L_0 - k.x,
\end{equation}
where k depends on the humidity, temperature, pressure as well as the molecular composition of atmosphere and is proportional to the square of the frequency. Thus, higher frequencies are absorbed by a far greater magnitude. This causes air to act as a low-pass filter over large distances. Molecular relaxation \cite{bass1990atmospheric}, \cite{evans1972atmospheric} is an important factor and losses due to oxygen-water vapour molecular relaxation are predominant above 500Hz. The absorption due to this factor is atleast 2 dB/kilometer irrespective of humidity and increases rapidly with frequency. The total absorption below 200 Hz is less than 1 dB/kilometer and decreases with frequency. If the air is extremely dry ($< 10\%$ relative humidity), the oxygen-carbon dioxide relaxation becomes significant and causes an almost constant absorption down from 500Hz to 80Hz of around 2dB/kilometer. The total air absorption as a function of frequency can be seen in Fig. \ref{Fig:airAbsorption}.
\subsubsection{Diffraction and barriers}
Barriers are sometimes purposefully built to block the direct path from the sound source to the receiver. Sound reaches the receiver either going through the barrier or by diffracting around the top of the barrier. Ground reflections and multi-path-propagation may lead to multiple diffracted wave paths. For a barrier, the ISO 9613-2 \cite{ISO9613} provides the following equation for loss due to barrier insertion
\begin{equation}
    IL = 10log\bigg[3+\bigg(C_2\frac{\delta_1}{\lambda}\bigg)C_3K_{met}\bigg],
\end{equation}
where $\lambda=wavelength$. The value of $C_2$ determines if ground reflections are taken care of ($C_2 = 20$) or not ($C_2 = 40$), $C_3$ is a factor to take care of double diffraction due to a barrier of finite thickness (or two thin barriers placed some distance apart), $\delta_1$ is the difference in distance between the direct source-to-receiver path and the wave propagation path caused by the barrier, and $K_{met}$ is a correction factor for average downwind meteorological effects. For thin barriers the equation simplifies to 
\begin{equation}
    IL = 10log(3+40\frac{\delta_1}{\lambda}). 
\end{equation}
Over large distances even buildings act like barriers, with the rooftop causing double diffraction. ISO 9613-2 \cite{ISO9613} provides a simple empirical method to calculate attenuation due to buildings.
\subsubsection{Ground effects}
When both source and receiver are close to the ground, interference of sound travelling directly from  source-to-receiver and sound reflected from the ground causes various ground effects. This interference can of course be both constructive or destructive. On acoustically hard surfaces such as non-porous asphalt or concrete, ground effects cause sound pressure to approximately double across a wide range of frequency. For porous surfaces, lower frequencies are enhanced while the higher frequencies get absorbed by the ground. For plane waves, the reflection coefficient of sound waves reflecting from the ground at angle $\phi$ is given by
\begin{equation}
R = \frac{sin (\phi) - Z_1/Z_2}{sin (\phi) + Z_1/Z_2},
\end{equation}
the $\phi \approx 0$. $Z_1$ and $Z_2$ are acoustical impedance of air and ground respectively. For infinitely hard surfaces $Z_1/Z_2 = 0$ and $R \to 1$. For infinitely soft surfaces  $Z_1/Z_2 = \infty$ and $R \to -1$. This means that there is a phase change upon reflection on acoustically soft surfaces, which causes destructive interference and can be seen as ground absorption. Of course the impedance itself is a function of frequency, so what acts as a relatively hard surface for low frequencies can act as a relatively soft surface for higher frequencies. 

Note that for large distances, $\phi \to 0$ (grazing incidence) and thus $R \to -1$, which predicts a net zero field over large distances irrespective of the values of $Z_1$ and $Z_2$. The plane wavefront assumption is the cause of this obvious error. Assuming spherical waves and grazing incidence, the equation for pressure at a distance r becomes

MORE ON THIS SOON!!!
\chapter{Propagation model}
Various methods have been used \cite{benesty2008microphone} to solve the estimation problem discussed above. The estimation problem would be trivial if the source signals received at each of the different microphones were simply delayed and scaled versions of each other. However, the source signal at each microphone is generally intermixed with surrounding noise. On top of that it contains attenuated and distorted copies of itself due to boundary reflections (if the measurement is not being made in free field or its approximation like an anechoic chamber). Add to this the fact that the source itself might be moving leading to a changing TDOA to the different microphones. All these factors quickly convert TDOA estimation to a non-trivial problem. \\
Based on these factors, the problem can be solved for assuming a variety of acoustical models. Beginning for a single source in a free-field (non-reverberant), to a single-source in a reverberant environment, to multiple-sources in a free-field or reverberant environments, and lastly, multiple-moving-sources. The following section will formulate each of these models.

Suppose our sources are in a sound field with an array of N microphones. Then the different possible acoustic models for TDOA analysis can be formulated as below.

\section{Single-source in free-field}
Let the acoustic signal from the single-source be \textit{s(k)} at time k. Then the signal received by the $n^{th}$ microphone at time k can be given be divided into the signal $x_n(k)$ and noise $\eta_n(k)$
\begin{equation}
    \begin{split}
    y_n(k) &= x_n(k) + \eta_n(k),\hspace{10pt} n = 1,2,....,N\\ 
           &=\alpha_n s(k-t-\tau_{n1}) + \eta_n(k),
    \end{split}
\end{equation}
where $\alpha_n$ is the gain/ attenuation of the signal at microphone n, t is the delay to the first microphone that receives the signal and $\tau_{n1}$ is the relative time delay of receiving signal between the first and the $n^{th}$ microphone, $\eta_n(k)$ being the noise at the $n^{th}$ microphone at time k. $\tau_{n1}$ can be defined as, $\tau_{n1}=\zeta(\tau)$, where, $\tau$ is the time delay between a pre-determined pair of microphones (say, microphone 1 and 2). Given $\tau$ and knowing the array geometry, it is possible to determine possible solutions for the rest of the delays between microphone pairs. Of course, there might not be a unique solution given a single value of $\tau$. $\zeta(\tau)$ can then be defined as a function of $\tau$ which can linear or a higher order polynomial depending on the array geometry. The estimation problem is then that of determining the estimate $\hat{\tau}$ of the true time delay $\tau$, given a finite number of observations.

\section{Multiple-sources in free-field}

For multiple sources (m = 1,2,....M), the signal received at $n^{th}$ microphone can be defined as : 
\begin{equation}
    \begin{split}
         y_n(k) &= x_n(k) + \eta_n(k),\hspace{10pt} n = 1,2,....,N \\
                &= \sum\limits_{m=1}^M \alpha_{nm}s_m[k - t_m - \zeta({\tau_m})] + \eta_n(k),
    \end{split}
\end{equation}
which is similar to the equation for the single-source, except the gain factor and the signal now depend on the $m^{th}$ source, so the total signal received at $n^{th}$ microphone is the sum of the signals from all M sources. The estimation problem is then that of determining all the estimates $\hat{\tau}_m$ of the true time delays $\tau_m$, given a finite number of observations.

\section{Single-source in reverberant-field}

For a single source in a reverberant field, finding the delay is not a simple problem. A reverberant field can be represented by a sufficiently long FIR filter, which adds multiple delayed and attenuated version of the source signal to the received signal, the number depending on the number of taps on that filter. The signal received at the $n^{th}$ microphone can be given by
\begin{equation}
    \begin{split}
         y_n(k) &= x_n(k) + \eta_n(k),\hspace{10pt} n = 1,2,....,N \\
                &= g_n*s(k) + \eta_n(k),
    \end{split}
\end{equation}
where $g_n$ is the impulse response from the source to the $n^{th}$ microphone. 
The $g_n$ then signifies a filter which adds multiple attenuated copies of the source signal at the receiver due to reverberation and also the actual source signal delay. So, the true delay for the same source signal on different microphone is hidden in this filter. It can be thought of as a multiple source problem, but all the sources are playing the same signal. The problem then is of separating these multiple sources with the knowledge of the reverberant field. If the goal is room auralization then the reverberation needs to be captured as image sources, however if the goal is source localization, then a technique needs to be devised so that the reverberation effect is filtered converting the problem to an approximation of the single-source in a free field model. 

\section{Multiple-sources in reverberant-field}
More on this soon!!
%\input{2x2_Tetrahedral_Array}
\section{Methods}
\subsection{Generalized correlation method}
The famous Knapp-Carter paper details the generalized cross-correlation method (GCC) for estimation of time delay \cite{1162830} in free field. For a pair of microphones, $m_1$ \& $m_2$, separated by a distance, the signals from a source received at time t can be given by
\begin{equation}
    \begin{split}
        x_1(t) &= s_1(t) + n_1(t) \\
        x_2(t) &= \alpha s_1(t - D) + n_2(t) ,
    \end{split}
\end{equation}
where $n_1(t)$ \& $n_2(t)$ are the noise at time t at the two microphones which are uncorrelated to the signal $s_1(t)$. The microphone $m_1$ receives the signal $s_1(t)$ first, while the microphone $m_2$ receives a delayed and attenuated version $\alpha s_1(t - D)$ at time t. The $\alpha$ depends on the microphone relative distance and microphone calibration and within-media factors like absorption. The time delay D depends on the microphone pair relative distance, the speed of sound in the media and the position of the sound source. 

Based on the discussions in previous sections, if we can estimate the value of D, we can estimate the source location. However, depending on source movement and environmental factors, both $\alpha$ and D can change over time. The estimation of D thus can only be made for observations of a finite duration. D can be estimated by computing the cross-correlation of the two signals
\begin{equation}
        R_{x_1x_2}(\tau) = \textbf{E[}x_1(t)x_2(t+\tau)\textbf], 
\end{equation}
Assuming noise to be uncorrelated to each other as well as the source signal, the cross correlation can be expressed as
\begin{equation}
    \begin{split}
        R_{x_1x_2}(\tau) &= \textbf{E[}\{s_1(t) + n_1(t)\}\{\alpha s_1(t+\tau - D) + n_2(t)\}\textbf] \\
                         &= \alpha\textbf{E[}s_1(t)s_1(t+\tau - D)\textbf] \\
                         &= \alpha R_{s_1s_1}(\tau - D),
    \end{split}
    \label{Eq:crosscorr}
\end{equation}
this cross-correlation peaks at $\tau - D = 0$, i.e. $\tau = D$. So the $\tau$ that maximizes the cross-correlation is an estimator for the time delay D. Assuming the processes to be ergodic so that the samples from a finite duration T can be used to estimate the cross-correlation, the estimate can be given by
\begin{equation}
    \hat{R}_{x_1 x_2}(\tau) = \frac{1}{T-\tau}\int_{0}^{T-\tau}x_1(t)x_2(t+\tau)dt,
\end{equation}
choosing sample mean as the estimator. Notice that even though the observation interval is T, we can only get usable information for time T - $\tau$, as we will have no corresponding signal for microphone $m_2$ for any signal that we receive at microphone $m_1$ after that time.

Taking the Fourier transform of Eq. \ref{Eq:crosscorr} to move to the frequency domain
\begin{equation}
        G_{x_1x_2}(f) = \alpha G_{s_1s_1}(f)\cdot e^{-j2\pi fD},
        \label{Eq:Gx1x2Gs1s1}
\end{equation}
multiplication with $e^{-j2\pi fD}$ in the frequency domain becomes convolution in the time domain
\begin{equation}
        R_{x_1x_2}(\tau) = \alpha R_{s_1s_1}(\tau) \circledast\delta(\tau - D),
        \label{Eq:Rx1x2}
\end{equation}
which can be seen as the Fourier transform of the signal spectrum spreading the delta function. The way to ensure no spreading takes place is to use a white noise signal. The autocorrelation of white noise is a delta function, in which case convolution with the delay-delta function results in a single peak value. Of course, in any kind of a reverberant field this will never be a single value. This is because the reverberations will have the effect of making the signal add up in a periodic and attenuated manner. However, the peak of the autocorrelation $R_{x_1x_2}(\tau)$ still happens at $\tau = D$, with the spreading having the effect of broadening the peak. If the time delay D is not a single value however, as can be the case in reverberant fields or for periodic signals, the $R_{x_1x_2}(\tau)$ will have multiple peaks. Each broad peak will overlap with the other in an additive or destructive manner making is impossible to detect or distinguish peaks. 

$R_{s_1s_1}(\tau)$ in Eq. \ref{Eq:Rx1x2} can be expanded to frequency domain to get
\begin{equation}
        R_{x_1x_2}(\tau) =  {\bigg[\int_{-\infty}^{\infty}\alpha{G}_{s_1s_1}(f) e^{-j2\pi f\tau} df\bigg]} \circledast\delta(\tau - D),
\end{equation}
this cross-correlation $R_{x_1x_2}(\tau)$ is a function that is spread around $\delta(\tau - D)$ according to ${G}_{s_1s_1}(f)$. This spreading is detrimental to the resolution of the localization results. Also, if the signal itself is non-stationary, like speech signals, this spreading is also unpredictable.

Now we are ready to form a basis for the different GCC weighing methods. If \textit{a priori} signal or noise information is available, the signals  $x_1(t)$ \& $x_2(t)$ can be pre-filtered to improve the accuracy of estimating the time delay. The method of selection of the pre-filter weights then forms the basis for the different GCC methods. 

Suppose, $x_1(t)$ \& $x_2(t)$ are filtered through filters $H_1(f)$ and $H_2(f)$, to get filtered signals $y_1(t)$ \& $y_2(t)$ respectively, then we have
\begin{equation}
        G_{y_1y_2}(f) = H_1(f)H^*_2(f) G_{x_1x_2}(f),
\end{equation}
taking the Fourier transform
\begin{equation}
\begin{split}
            R_{y_1y_2}(\tau) &= \int_{-\infty}^{\infty}H_1(f)H^*_2(f) G_{x_1x_2}(f) e^{-j2\pi f\tau} df \\
                             &= \int_{-\infty}^{\infty}\psi(f) G_{x_1x_2}(f) e^{-j2\pi f\tau} df,
\end{split}
\end{equation}
where 
\begin{equation}
            \psi(f) = H_1(f)H^*_2(f),
\end{equation}
since we can only estimate the cross-power spectra, we can write
\begin{equation}
            \hat{R}_{y_1y_2}(\tau) = \int_{-\infty}^{\infty}\psi(f) \hat{G}_{x_1x_2}(f) e^{-j2\pi f\tau} df,
\end{equation}
the frequency weights given by $\psi(f)$ can be selected according to the purpose that is wished to be achieved. For example, if the purpose is to maximize the signal-to-noise (SNR) ratio in the signal passed, then the $\psi(f)$ could be selected so that it attenuates the frequencies in the noise spectra. Obviously this requires either priori-knowledge or estimation of the noise spectra. The following sections introduce the different methods of frequency weight selection. Four methods are described here, ROTH, SCOT, PHAT and ML. Of particular interest are the PHAT and ML, direct and improved versions of which have been consistently used to do robust source localization. 

\subsubsection{ROTH}
The frequency weights for ROTH processor are defined as
\begin{equation}
            \psi(f) = \frac{1}{G_{x_1x_1}(f)},
\end{equation}
so we get 
\begin{equation}
            {R}_{y_1y_2}(\tau) = \int_{-\infty}^{\infty}\frac{{G}_{x_1x_2}(f)}{G_{x_1x_1}(f)} e^{j2\pi f\tau} df,
\end{equation}
substituting the value for ${G}_{x_1x_2}(f)$ assuming uncorrelated noise from Eq. \ref{Eq:Gx1x2Gs1s1} we get
\begin{equation}
\begin{split}
                \hat{R}_{y_1y_2}(\tau) &= \int_{-\infty}^{\infty}\frac{\alpha\hat{G}_{s_1s_1}(f)}{G_{x_1x_1}(f)} e^{j2\pi f.(\tau-D)} df \\
                                        &= \delta (\tau - D) \circledast \bigg[\int_{-\infty}^{\infty}\frac{\alpha\hat{G}_{s_1s_1}(f)}{G_{s_1s_1}(f) + G_{n_1n_1}(f)} e^{j2\pi f\tau}  df\bigg],
                                        %&= \delta (\tau - D) \circledast \alpha\bigg[\int_{-\infty}^{\infty}\frac{{G}_{s_1s_1}(f) + G_{n_1n_1}(f) -G_{n_1n_1}(f)}{G_{s_1s_1}(f) + G_{n_1n_1}(f)} e^{j2\pi f\tau}  df\bigg] \\
                                        %&= \delta (\tau - D) \circledast \alpha\bigg[\int_{-\infty}^{\infty}e^{j2\pi f\tau}df-\int_{-\infty}^{\infty}\frac{G_{n_1n_1}(f)}{G_{s_1s_1}(f) + G_{n_1n_1}(f)} e^{j2\pi f\tau}  df\bigg] \\
\end{split}
\end{equation}
so now the delta function is spread according to the value of $G_{n_1n_1}(f)$. For frequencies f where $G_{n_1n_1}(f)$ has a high magnitude, the cross-correlation will be suppressed, so that peaks in the frequency regions where $n_1$ is high disappear. But as can be seen ROTH processor does nothing to improve the high $n_2$ regions or the spreading around the main peak.

\subsubsection{SCOT}
The frequency weights for SCOT processor are defined as
\begin{equation}
            \psi(f) = \frac{1}{\sqrt{G_{x_1x_1}(f)G_{x_2x_2}(f)}},
\end{equation}
so this takes care of regions where either $n_1$ or $n_2$ might be high solving a possible disadvantage with ROTH. 

\subsubsection{PHAT}
Both SCOT and ROTH suffer from the disadvantage that the value of ${R}_{y_1y_2}(\tau)$ is spread around the delta function depending on the cross-spectrum ${G}_{x_1x_2}(f)$. However, the TDOA information is carried only by the phase of the cross-spectrum and not the amplitude. So, setting the weights as
\begin{equation}
            \psi(f) = \frac{1}{|G_{x_1x_2}(f)|},
\end{equation}
we get
\begin{equation}
        {R}_{y_1y_2}(\tau) = \int_{-\infty}^{\infty}\frac{{G}_{x_1x_2}(f)}{|G_{x_1x_2}(f)|}e^{j2\pi f\tau}   df,
\end{equation}
Now, we have from Eq. \ref{Eq:Gx1x2Gs1s1}
\begin{equation}
\begin{split}
        |G_{x_1x_2}(f)| &= \alpha G_{s_1s_1}(f) \\
    \frac{{G}_{x_1x_2}(f)}{|G_{x_1x_2}(f)|}&=e^{-j2\pi fD},
\end{split}
\label{Eq:ModGx1x2}
\end{equation}
where the magnitude information is cancelled and only the phase information remains, where D is the delay or the 'phase'. We get
\begin{equation}
\begin{split}
            {R}_{y_1y_2}(\tau) &= \int_{-\infty}^{\infty}e^{j2\pi f(\tau - D)}   df \\
                           &=  \delta (\tau - D) \circledast \int_{-\infty}^{\infty}e^{j2\pi f\tau} df,
\end{split}
\label{Eq:RY1Y2}
\end{equation}
So ideally PHAT weighing gives a cross-correlation value that has no spreading and gives a clean peak at $\tau=D$.

Even though PHAT seems to solve all problems, the method is not without issues. Most of the issues arise from the assumptions made for PHAT. These are itemized below: 
\begin{itemize}
    \item $n_1$ and $n_2$ are assumed to be uncorrelated. If that is not the case, the magnitude of $G_{x_1x_2}(f)$ would obviously not cancel out in Eq. \ref{Eq:ModGx1x2}
    \item The GCC methods assume single-source in free-field model, ie, no reverberation is assumed. The effects of reverberation can actually be moderate to severe in PHAT and have been discussed in various papers [Cite papers here]. 
    \item The 'expected' value of $G_{x_1x_2}(f)$ is assumed to be known. In reality it can only be estimated, viz $\hat{G}_{x_1x_2}(f)$. In situations where $\hat{G}_{x_1x_2}(f) \neq G_{x_1x_2}(f)$, the cross-correlation in Eq. \ref{Eq:RY1Y2} will not be a delta function. This error is magnified even more in regions where $G_{s_1s_1}(f)$ is very low. This has the potential to cause PHAT to provide poor results in low SNR conditions.
\end{itemize}  

A practical issue that exists with GCC methods is the angular resolution of localization. If the signals are recorded at 44.1kHz sample rate, then the minimum time delay allowed is $1/$ 44100 sec for 1 sample delay. For 2 microphones placed distance $20$ cm apart, the minimum resolution achievable in this time is $2.2\degree$ broadside to $16\degree$ endside, assuming speed of sound to be 343 m/sec (Fig. \ref{fig:ang_res}). At 192kHz and 1m microphone distance, the issue is less severe, being $0.5\degree$ broadside to $7.7\degree$ endside.
\begin{figure}[H]
     \centering
     \includegraphics[width=0.8\textwidth]{Figures/AngularRes.png}
     \caption{Figure represents plane wave incidence on a microphone pair. For broadside incidence the time delay is the minimum = 0 between the two microphones. The next time delay allowed is $1/$fs, corresponding to travel distance of $c/$fs (c being the speed of sound). So we have  dsin($\theta$) = $c$/fs. For endside incidence the time delay is maximum = $d/c$. The next time delay allowed is $d/c$-$1/$fs, corresponding to travel distance of $d$-$c/$fs, and we have dsin($\theta$) =  $d$-$c/$fs.}
     \label{fig:ang_res}
\end{figure}

The issue is solved by curve fitting and interpolation. Parabolic curve fitting was initially proposed method to solve it, but was shown to be a biased estimator \cite{boucher1981analysis}. Consequently, various interpolation techniques have been developed to overcome this issue \cite{jacovitti1993discrete}, \cite{brandstein1997practical}, \cite{zhang2005cross}, \cite{tervo2008interpolation}. The 2D localization resolution with no interpolation is plotted in Fig. \ref{fig:res_diff}.

\begin{figure}[H]
    \centering
    \includegraphics[width=\textwidth]{Figures/res_diff.png}
    \caption{Frontal 2D localization resolution for different sample rates and distance between a pair of microphones. As can be seen large apertures and high sample rates have a better resolution than lower sample rates and smaller apertures.}
    \label{fig:res_diff}
\end{figure}

Some simulations for GCC are given in Fig. \ref{fig:GCC_SIM}. It can be seen that the resolution falls the closer we get to end-side ($0\degree$ and $180\degree$). Also it can be seen that the results are poor if no weights are used. PHAT and SCOT perform quite similarly in the simulations, with PHAT being marginally better. It can be seen that the level difference is maintained between the 2 sources in the results. However no peaks are visible if the SNR falls to 0 dB.

\begin{figure}[h]
    \centering
    \begin{subfigure}[b]{0.48\textwidth}
    \centering
    \includegraphics[width=0.9\textwidth]{Figures/GCC_40.png}
    \caption{Both sources at 40dB SNR}
    \label{fig:d1}
\end{subfigure}
\hfill
\begin{subfigure}[b]{0.48\textwidth}
    \centering
    \includegraphics[width=0.9\textwidth]{Figures/GCC_20_40.png}
    \caption{S1 at 20dB SNR, S2 at 40dB SNR}
    \label{fig:d2}
\end{subfigure}
\vskip \baselineskip
\begin{subfigure}[b]{0.48\textwidth}
    \centering
    \includegraphics[width=0.9\textwidth]{Figures/GCC_20_20.png}
    \caption{Both sources at 20dB SNR}
    \label{fig:d3}
\end{subfigure}
\quad
\begin{subfigure}[b]{0.48\textwidth}
    \centering
    \includegraphics[width=0.9\textwidth]{Figures/GCC_0_20.png}
    \caption{S1 at 20 SNR, S2 at 0 SNR}
    \label{fig:d4}
\end{subfigure}
\caption{Figures compares different GCC algorithms for localization performance for 2 sources with various SNRs. The simulations assume 2 microphones placed 1m apart along the $0\degree-180\degree$ axis. The sampling rate is assumed to be 192kHz and speed of sound is 343m/sec. Two sources playing pink noise at different levels and located at $S_1:15\degree$ and $S_2:100\degree$ are assumed. Uncorrelated white noise is assumed to be present at the 2 microphones. No interpolation fixing is done. The level of the noise is unchanged but the level of the signal is varied to achieve the different SNRs.}
\label{fig:GCC_SIM}
\end{figure}


\subsection{Multiple pair GCC}

GCC equations described above are for a single pair of microphones only. Various algorithms have been designed that extend the GCC algorithm to multiple pairs of microphones. SRP-PHAT approach \cite{dibiase2000high} combines the steered response power (SRP) beamformer methods \cite{krim1996two} to the GCC approach. Griebel \cite{griebel2001microphone} describes a method where the \enquote{GCC functions derived from various microphone pairs are simultaneously maximized over a set of potential delay combinations consistent with candidate locations} which can be seen as a special case of SRP-PHAT where the redundant information from additional microphone pairs are utilized. \textit{Okuyama et al.} show in a 2002 study\cite{okuyama2002study} that when using a spatial array like a tetrahedron, the propagation direction of sound through the array can be determined, irrespective of the speed of sound, by using the least-squares approach. This means that for localizing sound sources outdoors, the instantaneous temperature and wind on the microphone array need not be known. Benesty \cite{benesty2004time} provides a method to fully utilize the redundant information from multiple microphone pairs to make the time-delay estimation (TDE) process more robust against distortion and also improve angular resolution. The method re-derives multi-channel cross correlation (MCCC) to apply linear interpolation on the GCC data to improve the angular resolution of localization. More recently, in \cite{liu2010continuous} the author used a motorized robot with 4 microphone arranged in a cross-formation. The algorithm uses 'de-noising' techniques such as adding a small regularization term to the denominator of the PHAT weight, which can reduce the low SNR issues surrounding PHAT. The low SNR regions can be further penalized by using reliability-weighted RW-PHAT \cite{valin2006robust}, where a-priori SNR information is used to estimate the weight to be multiplied during the PHAT computation. Hu\cite{hu2009estimation} provides a method to do eigenvalue decomposition based GCC (ES-GCC) with ES-GCC producing lesser number of outlier locations that GCC-PHAT. Badali \cite{badali2009evaluating} compares various localization algorithms using a 8 microphone array located on a cube. The authors use hyperbolic intersection on the GCC results from multiple pairs of microphones. They conclude that if \textit{Direction Refinement} procedure is run, in which first a far-field assumption search is done and the locations are then `refined' for near field, then the results from SRP-PHAT can be improved. But this procedure might not be relevant for far-field outdoor localization.  

The next sections will discuss some of the algorithms that are relevant for outdoor source localization and make the PHAT process more robust. 
\section{SRP-PHAT}\label{sec:SRP}
Steered Response Power (SRP) source localization is a method to detect sound source locations using beamforming techniques \cite{krim1996two}. SRP is different from TDOA based methods discussed before. While the generalized cross correlation is a simple cross correlation between each pair of microphones and outputs an estimate of the time delay, the SRP method beamforms the space around the array and computes the energy of each location beam.  It `looks' at all possible directions individually (steering) and computes the power of the signal cross correlation in that direction (beamforming). The assumption is that the cross power of the steered microphone array will be the maximum in the correct source direction. However, the computational demand for this can rise quite fast (depending on the sample rate and the angular resolution of the beamforming), making it nearly impossible to implement in real time applications. But, its performance in difficult conditions outperforms the TDOA based methods \cite{dmochowski2007generalized}. Since real-time localization is not of primary importance for this thesis, SRP based methods can be applied. However, in the same fashion as the GCC methods proposed to pre-filter the signal before performing the cross correlation, PHAT weighing can also be applied on the beamformed signal. The method, called SRP-PHAT, combines the robustness of the SRP to the accuracy of the PHAT. 
\subsection{Steered response power}
The SRP method is based on a regular delay-and-sum beamformer, for a given point in space having range $\rho$, azimuth $\theta$ and elevation $\phi$ with the microphone array, the output of the beamformer is given by
\begin{equation}
    y_{\rho,\theta,\phi}(n)=\sum\limits_{m=0}^{M-1}{w_m x_m[n + f_{0,m}(\rho,\theta,\phi)]},
\end{equation}
where $x_0[n]$ is the signal received at time n, at an arbitrary microphone used as reference, $w_m$ is the amplitude weight for microphone m, and $f_{0,m}(\rho,\theta,\phi)$ is the relative delay between the reference microphone and the $m^{th}$ microphone. When far-field approximation is assumed, the range cannot be computed\footnote{For range computation, the cone approximation cannot be assumed. The delays should be used to compute hyperboloids and not cones. The intersection of the hyperboloids can then be used to compute range. However, it should be remembered that even a small error would lead to large variations in range, as for far-field, small movements in the hyperboloids would cause large movements in range results.} and the delay-and-sum beamformer output can be rewritten as follows:
\begin{equation}
    y_{\theta,\phi}(n)=\sum\limits_{m=0}^{M-1}{w_m x_m[n + f_{0,m}(\theta,\phi)]},
\end{equation}
For $w_m=1$ (assuming perfectly omni-directional and equally sensitive microphones), the output power of the beamformer becomes
\begin{equation}
    \mathbb{E}[{y_{\theta,\phi}(n)^2}]=\sum\limits_{i=0}^{M-1}\sum\limits_{j=0}^{M-1}{R_{x_i,x_j}[f_{i,j}(\theta,\phi)]} \text{, for } i\neq j.
    \label{eq:poweroutputbeamformer}
\end{equation}
This cross correlation is computed in the frequency domain (cross-spectrum) which is then inverse fast Fourier transformed (IFFT).
\begin{equation}
    R_{x_i,x_j}(\tau)= \sum\limits_{k=0}^{N_{f}-1}{X_{i}(k)X_{j}^*(k)e^{j2\pi\frac{k}{N_{f}}\tau}}
\end{equation}

Where $X_{i}(k)$ is the $N_{f}$ (number  point Fast Fourier Transform of a signal from the $i$ microphone.  

\subsection{SRP algorithm}
\begin{itemize}
    \item Compute the cross correlations of the signals received at all the microphone pairs. 
    \item Compute for each set of angles ($\phi,\theta$), the corresponding set of delays for every microphone pair $f_{i,j}(\theta,\phi)$. So if 1$\degree$ angular resolution is used, the SRP method computes delays for 360*180=64800 angular positions, for each microphone pair.
    \item For each ($\phi,\theta$), sum for all microphone pairs, the cross correlation values at the corresponding delays. This sum is the output of the SRP beamformer defined in Eq. \ref{eq:poweroutputbeamformer}\footnote{An improvement on the SRP search algorithm was proposed by pre-mapping the relative delays to their corresponding set of locations \cite{dmochowski2007generalized}. Instead of proceeding with a full sequential search in the 3D space, a search on the possible relative delays, where the cross correlation values are above a threshold, is considered. The possible delays between individual microphone pairs are already known based on the array geometry and can be stored in memory. The computational cost gain can be immense depending on the number of microphones. However, the method is not suitable if the whole acoustic map of an environment in required, so it is not detailed further here.}

%The cross correlations are calculated for each delay subset and related to a set of potential source location in space in the final steps of the algorithm. 
\end{itemize}
\begin{equation}
    S_{SRP}(\theta,\phi)=\sum\limits_{i=0}^{M-1}\sum\limits_{j=0}^{M-1}{R_{x_i,x_j}[f_{i,j}(\theta,\phi)]} \text{, for } i\neq j.\footnote{For a single source, the estimate of the source location from SRP search can be given by:
    $\hat{\phi},\hat{\theta}=\argmax_{\phi,\theta}S_{SRP}(\phi,\theta)$}
     \label{eq:srpSum}
\end{equation}
%\begin{figure}
%    \centering
%    \begin{subfigure}[t]{0.5\textwidth}
%    \centering
%    \includegraphics[width=0.9\textwidth]{Figures/viewside.png}
%    \caption{SRP map}
%    \label{fig:viewsidesrp}
%\end{subfigure}%
%\begin{subfigure}[t]{0.5\textwidth}
%    \centering
%    \includegraphics[width=0.9\textwidth]{Figures/topview.png}
%    \caption{SRP heat map}
%    \label{fig:topviewsrp}
%\end{subfigure}
%\caption{Power mapping simulation of source localized using SRP-PHAT algorithm in noiseless, free-field situation. For the simulation, first a 4-channel wav file is created such that the channels contain the same pink noise but delayed between each other. The delays are such that the source would be ideally located at azimuth $80\degree$ and elevation $40\degree$ when localized by a 1m aperture tetrahedral array. SRP-PHAT is then applied on the wav file and power received from different angles (beams) is computed and plotted. As can be seen the algorithm was able to localize the source in these ideal conditions fairly correctly.}
%\end{figure}
%The classical SRP search beamforms sequentially the 3D space and locations [($\phi_{1},\theta_{1}$), ($\phi_{2},\theta_{2}$), ... ,($\phi_{x},\theta_{x}$)] which might be associated with the same relative delay $\tau_{1}$ (in case of a uniform linear microphone array). The cross correlation at delay $\tau_{1}$ is then computed $x$ times, which leads to the same results for each [($\phi_{1},\theta_{1}$), ($\phi_{2},\theta_{2}$), ... ,($\phi_{x},\theta_{x}$)] positions, leading to numerous useless cross correlation computations.
%In GCC methods this issue was taken care of by interpolation, where the microphone pair end-side localization had poor resolution (Fig. \ref{fig:res_diff}). 
\subsection{Extending PHAT to SRP-PHAT}
PHAT can be extended to SRP-PHAT, by simply pre-filtering the cross-correlations before the SRP sum step,
\begin{equation}
    R_{x_i,x_j}(\tau)= \sum\limits_{k=0}^{N_{f}-1}{\psi_{ij}(k) X_{i}(k)X_{j}^*(k)e^{j2\pi\frac{k}{N_{f}}\tau}}
\end{equation}
where
\begin{equation}
    \psi_{ij}(k) = \frac{1}{|{X_{i}(k)X_{j}^*(k)}|}
\end{equation}
%Beamforming techniques for sound localization have been study intensively over the last decades. The main drawback of the conventional beamforming are the side lobes in the localization results. 
% the ideal result is to detect a point precisely at the actual point source location and nothing elsewhere. However that is not the case for SRP-PHAT. For instance, i
\subsection{Localizing with SRP-PHAT}
When localizing a point source using SRP-PHAT, if two microphones are used, the only information that can be computed is the angle of incidence of the source on the array. For example, for a source located at (-50$\degree$, 60$\degree$)\footnote{For the purpose of this thesis, locations are designated as (x$\degree$, y$\degree$), signifying (azimuth, elevation) of the location, respectively, in spherical coordinates.}, the result is a circle around the array where the source might be located, shown in Fig. \ref{fig:2mic1src}. This is because the angle of incidence from every point on the circle, to the mid point of the line joining the two microphones, is the same. This circle is actually the base of the hyperboloid discussed in Sec. \ref{sec:TDOA}.
%For far-field, this means a plane wave incident with a particular DOA on the microphone array. In transfer function terms, the array response is `deconvolved' from the final energy map. 
%While this problem has motivated the creation of new beamformers {!!CITATION!!}, the problem lies in the method itself.
%New classes of algorithms have been developed to deconvolve the noise signals from the desired steered signal such as CLEAN \cite{sijtsma2007clean} and DAMAS \cite{brooks2006deconvolution}.
%DAMAS was acknowledged a major breakthrough in array processing. At first research was mainly focused on aeroacoustic for the development of near-field sound localization system but it seems that a new enthusiasm has taken over scientists trying to solve other sound localization problems. 
%While the DAMAS method is mainly designed for near field measurements in the range of the array aperture size, a new deconvolution method has been proposed [\cite{zhao2015large}, \cite{zhao2017large}] where a small aperture array is used to measure source signal in the far field. The principle behind point spread function is discussed in the following section. Then a review of the underlying principles of the main deconvolution algorithms is given, and finally a specific method for coherent and incoherent sources localization is discussed.

%\begin{figure}[H]
%    \centering
%    \includegraphics[width=0.98\textwidth]{Figures/2mic1src.png}
%    \caption{Figure depicts a source located at $50\degree$ azimuth and $60\degree$ elevation (orange dot). Two microphones (blue dots) will be used to localize the source.}
%    \label{fig:2mic1srcPos}
%\end{figure}
\begin{figure}[H]
    \centering
    \includegraphics[width=0.8\textwidth]{Figures/2mic1srcRes.png}
    \caption{SRP-PHAT is run to localize a single point source using two microphones $M_{1} and M_{2}$ as described in appendix \ref{app:micLocs}. The source can only be localized to a circle. The blue cross in the figure indicates the actual source location. Note that the reason the circle does not appear exactly circular in image is due to the cylindrical projection being used to display the result.}
    \label{fig:2mic1src}
\end{figure}
The results in Fig. \ref{fig:2mic1src} are displayed using the cylindrical projection technique, such that the entire spherical space around the origin can be shown as a rectangle, with x-axis being the azimuth and y-axis being the elevation. This technique is employed throughout the thesis to give a full picture of the localization results.

A new circle will result for each new microphone pair used to localize the source, as long as the microphone pairs are not all placed in the same line\footnote{In the case of a linear array, the multiple circles would overlap completely}. For example, for three microphones A, B and C placed in an equilateral triangle formation, three circles can be computed (one each for AB, BC and CA). The maximum peak occurs at 2 locations with (-50$\degree$,$\pm$ 60$\degree$) as shown in Fig. \ref{fig:3mic1src}. If a fourth microphone is placed in the same plane as the triangle, the array response will be a combination of circles from 6 possible microphone pairs ($^4C_2$). However, the new circles would all pass through the same 2 locations. For a non co-planar array, e.g. a tetrahedral array, the maximum peak occurs at exactly one poin, shown in \ref{fig:4mic1src}. 
\subsection{Some considerations with SRP-PHAT}
\subsubsection{Subsidiary peaks}
Even though a tetrahedral array is able to detect a point source to a single maximum peak position, subsidiary peaks can appear in the energy map at DOAs that don't correspond to the true source DOA, since, the cross-correlations values at computed delays are summed by the beamformer (Eq. \ref{eq:srpSum}). For example, points where only 2-5 of the circles meet. If multiple sources are localized, these peaks in the SRP-PHAT energy map can add up leading to the detection of a fake source and can also mask real sources. The effect of these subsidiary peaks can be reduced by increasing the number of microphones. This is because, even though more microphone pairs would mean more localization circles, it also means the real peaks would be higher, effectively lowering the noise/subsidiary peak floor. Indeed, solutions in the market exist with even 90 microphones (\cite{batel2003noise}, Fig. 31). However, since, the number of microphones is a constraint requirement for this thesis, this solution is not considered.
\subsubsection{Linear dependency}
Since only 3 pairs out of the 6 in a tetrahedral array are linearly independent (Eq. \ref{Eq:linearDep}), the localization can also be done considering only 3 of those pairs. The result in shown in Fig. \ref{fig:4mic1srcInd}.
\begin{figure}[!ht]
    \centering
    \includegraphics[width=0.8\textwidth]{Figures/3mic1srcRes.png}
    \caption{SRP-PHAT is run to localize the source with 3 microphones $M_{1}, M_{2} and  M_{3}$ as described in appendix \ref{app:micLocs}.}
    \label{fig:3mic1src}
\end{figure}
\begin{figure}[!ht]
    \centering
    \includegraphics[width=0.8\textwidth]{Figures/4mic1srcRes.png}
    \caption{SRP-PHAT is run to localize the source with a tetrahedral array ($M_{1}, M_{2},  M_{3} and M_{4}$ as described in appendix \ref{app:micLocs}).}
    \label{fig:4mic1src}
\end{figure}
\begin{figure}[!ht]
    \centering
    \includegraphics[width=0.8\textwidth]{Figures/Ind4mic1srcRes.png}
    \caption{SRP-PHAT is run to localize the source with a tetrahedral array but only linearly independent microphone pairs are considered}
    \label{fig:4mic1srcInd}
\end{figure}
Considering only independent microphones, Eq. \ref{eq:srpSum} can be rewritten as,
\begin{equation}
    S_{SRP}(\theta,\phi)=\sum\limits_{i=1}^{M-1}{R_{x_0,x_i}[f_{0,i}(\theta,\phi)]}
     \label{eq:srpSumInd}
\end{equation}
%\subsection{Using the redundant information from the microphone pairs}
The thing to note is that Eq. \ref{Eq:linearDep} is only true for no noise conditions. In noisy conditions, there is a potential to gain information by using the redundant microphone pairs. This is because if noise at all microphones is assumed to be uncorrelated, then even though the noise causes certain microphone pairs to detect a source at a `sourceless' location, other microphone pairs might detect a lower magnitude at that location. Due to more microphone pairs, the sum of all microphone pairs will be even higher at the real source location, and at other locations, the sum due to the noise will be suppressed. It can be seen in Fig. \ref{fig:4mic1src} that using all the microphone pairs adds to the overall noise on the map as more pairs can now contribute to the SRP sum, leading to more circles. However, the peak of the true source also becomes higher, due to more pairs providing power at the source location. This means that even though the noisy floor is has noise in more locations, it is of a lower magnitude, leading to a higher achievable dynamic range. For this reason, from here on the localization results considers all possible pair of microphones.

\section{Robust solution}

Beamforming techniques for sound localization have been study intensively over the last decades. The main drawback of the conventional beamforming are the side lobes in the localization results. If SRP-PHAT algorithm is applied to a tetrahedral array and cross-correlations values at different delays are summed by the beamformer (Eq. \ref{eq:srpSum}), subsidiary peaks can appear in the energy map at DOAs that don't correspond to the incident plane wave DOA. Those peaks in the SRP-PHAT energy map can mask real sources or even add up to other peaks from other sources, thus display a fake source. Deconvolution methods remove those peaks to reveal the correct peak. The algorithms for deconvolution are based on the point spread function, which is the response of the array to a point source. For far-field, this means a plane wave incident with a particular DOA on the microphone array. In transfer function terms, the array response is `deconvolved' from the final energy map. While this problem has motivated the creation of new beamformers {!!CITATION!!}, the problem lies in the method itself. New classes of algorithms has been developed to deconvolve the noise signals from the desired steered signal such as CLEAN \cite{sijtsma2007clean}. Even if CLEAN was applied successfully, it was in 2006 that the deconvolution approach for the mapping of acoustic sources (DAMAS) was proposed \cite{brooks2006deconvolution} and was acknowledged a major breakthrough in array processing. At first research was mainly focused on aeroacoustic for the development of near-field sound localization system but it seems that a new enthusiasm has taken over scientists trying to solve other sound localization problems. While the DAMAS method is mainly designed for near field measurements in the range of the array aperture size, a new method is proposed in \cite{zhao2015large} and \cite{zhao2017large} where a small aperture array is used to measure source signal in the far field. The principle behind point spread function is discussed in the following section. Then a review of the underlying principles of the main deconvolution algorithms is given, and finally a specific method for coherent and incoherent sources localization is discussed.

\subsection{Point spread function}
Point spread function is the response of an `imaging' system to a point source. In the context of sound localization, this is the loci of point sound source detected by the microphone array. A perfect array will detect a point precisely at the actual source location and nothing elsewhere. This, however, is not the general case. For example, given a single pair of microphones, and assuming far-field incidence, the only information that can be concluded from a single point source is the angle of incidence on the array. This angle is a vector combination of source azimuth and elevation. This leads to a circle around the array where the source might be located (circular maximum peak). This circle is the base of the cone resulting from the cone approximation discussed previously.

%\begin{figure}[H]
%    \centering
%    \includegraphics[width=0.98\textwidth]{Figures/2mic1src.png}
%    \caption{Figure depicts a source located at $50\degree$ azimuth and $60\degree$ elevation (orange dot). Two microphones (blue dots) will be used to localize the source.}
%    \label{fig:2mic1srcPos}
%\end{figure}

\begin{figure}[H]
    \centering
    \includegraphics[width=0.98\textwidth]{Figures/2mic1srcRes.png}
    \caption{SRP-PHAT is run to localize the source. As expected the source is localized to a circle. The red dot indicates the actual source location.}
    \label{fig:2mic1src}
\end{figure}

If three microphones are placed in a horizontal equilateral triangle, we get three circles (three possible pairs of microphones) from localization. The maximum peak occurs at 2 locations with azimuth=50$\degree$ and elevation=$\pm 60\degree$.
\begin{figure}[H]
    \centering
    \includegraphics[width=0.98\textwidth]{Figures/3mic1srcRes.png}
    \caption{SRP-PHAT is run to localize the source with 3 microphones.}
    \label{fig:3mic1src}
\end{figure}

For a tetrahedral array, the point spread function is a combination of circles from the 6 possible microphone pairs. This time the main peak occurs at exactly one point. 
\begin{figure}[H]
    \centering
    \includegraphics[width=0.98\textwidth]{Figures/4mic1srcRes.png}
    \caption{SRP-PHAT is run to localize the source with a tetrahedral array.}
    \label{fig:4mic1src}
\end{figure}

The result in fig.\ref{fig:4mic1src} is applicable only in ideal conditions (zero noise, no reflections and perfectly planar propagation). In noisy conditions, the localization result is depicted in fig. \ref{fig:4mic1srcNoisy}. Note that the color bars for the figures depicted are of the same color scale. The localization performance deteriorates as the SNR drops. The result is similar to the results for PHAT with 2 microphones discussed earlier as the underlying process in SRP-PHAT is still PHAT. 

As can be seen in Fig. \ref{fig:4mic1src}, even in ideal conditions, the localization results contain many peaks of varying heights. This is due to summing the cross-correlation responses of a non-linear array (Eq. \ref{eq:srpSum}). If the array were linear, the localization circles from each pair would all overlap completely. In case of a tetrahedral array, the localization circles from the possible microphone pairs are not co-planar. This is because all the edges of a tetrahedron point in the different directions. The obvious problem here is multi-source detection. If multiple sources are playing at different levels, how do we determine if the detected peak is a source or a relic from another higher level source? The methods to do so form the basis of deconvolution methods. 
A simple deconvolution approach could be to penalize sources detected only by a subset of the microphone pair combinations. This could be done by taking a product and not a sum in Eq. \ref{eq:srpSum}. This way, if a peak is caused by a single localization circle, the cross-correlation values from other microphone pairs would be close to zero, and thus would scale the false peak down. The localization results from this are given in fig. \ref{fig:4mic1srcNoisyProd}. Fig. \ref{fig:4mic2srcNoisyCompare} compares the localization result for two equally loud sources, this time located at (azimuth, elevation) = (-20, -30) and (50, 60), with summed SRP-PHAT and product-SRP-PHAT. The results are convincingly better, atleast for simulations.  

\begin{figure}[H]
    \centering
    \begin{subfigure}[b]{0.96\textwidth}
    \centering
    \includegraphics[width=0.8\textwidth]{Figures/4mic1src20.png}
\end{subfigure}
\vskip \baselineskip
\begin{subfigure}[b]{0.96\textwidth}
    \centering
    \includegraphics[width=0.8\textwidth]{Figures/4mic1src6.png}
\end{subfigure}
\vskip \baselineskip
\begin{subfigure}[b]{0.96\textwidth}
    \centering
    \includegraphics[width=0.8\textwidth]{Figures/4mic1src0.png}
\end{subfigure}
\vskip \baselineskip
\begin{subfigure}[b]{0.96\textwidth}
    \centering
    \includegraphics[width=0.8\textwidth]{Figures/4mic1srcNeg6.png}
\end{subfigure}
\caption{Figures depict from top-to-bottom SRP-PHAT localization results with SNR = 20dB, SNR = 6dB, SNR = 0dB, SNR = -6dB}
\label{fig:4mic1srcNoisy}
\end{figure}

 \begin{figure}[H]
    \centering
    \begin{subfigure}[b]{0.96\textwidth}
    \centering
    \includegraphics[width=0.8\textwidth]{Figures/4mic1src20Prod.png}
\end{subfigure}
\vskip \baselineskip
\begin{subfigure}[b]{0.96\textwidth}
    \centering
    \includegraphics[width=0.8\textwidth]{Figures/4mic1src6Prod.png}
\end{subfigure}
\vskip \baselineskip
\begin{subfigure}[b]{0.96\textwidth}
    \centering
    \includegraphics[width=0.8\textwidth]{Figures/4mic1src0Prod.png}
\end{subfigure}
\vskip \baselineskip
\begin{subfigure}[b]{0.96\textwidth}
    \centering
    \includegraphics[width=0.8\textwidth]{Figures/4mic1srcNeg6Prod.png}
\end{subfigure}
\caption{Figures depict from top-to-bottom product-SRP-PHAT localization results  with SNR = 20dB, SNR = 6dB, SNR = 0dB, SNR = -6dB}
\label{fig:4mic1srcNoisyProd}
\end{figure}

 \begin{figure}[H]
    \centering
    \begin{subfigure}[b]{0.96\textwidth}
    \centering
    \includegraphics[width=0.8\textwidth]{Figures/4mic2srcNeg6Sum.png}
\end{subfigure}
\vskip \baselineskip
\begin{subfigure}[b]{0.96\textwidth}
    \centering
    \includegraphics[width=0.8\textwidth]{Figures/4mic2srcNeg6Prod.png}
\end{subfigure}
\caption{Figures depict from localization results with summed-SRP-PHAT (top) and product-SRP-PHAT (bottom)}
\label{fig:4mic2srcNoisyCompare}
\end{figure}

The drawback of using product-SRP-PHAT is that the sound level difference between the different sound sources is lost. In summed-SRP-PHAT, the array magnitude response at a particular azimuth and elevation could be averaged over all microphone pair combinations. Then the level difference between 2 sources is maintained. In product-SRP-PHAT this would not be the case. However if it is assumed that a particular source will have similar magnitude response for all microphone pairs (which is not a strong assumption in far-field), then taking the root ($6^{th}$ root for a tetrahedral array) of the responses, the level difference can be maintained. When plotted on dB scale, that means a simple division by 6, which can be a disadvantage as can be seen in fig. \ref{4mic2srcNoisyDiv}.

\begin{figure}[h]
    \centering
    \includegraphics[width=0.98\textwidth]{Figures/4mic2srcNeg6ProdDiv.png}
    \caption{Product-SRP-PHAT is run to localize 2 sources located at (azimuth, elevation) = (-20, -30) and (50, 60) with a tetrahedral array. The result is divided by 6 and normalized before visualization to maintain the level difference between the different sources.}
    \label{fig:4mic2srcNoisyDiv}
\end{figure}
\subsection{Deconvolution of the array response}

As can be seen in fig. \ref{fig:4mic1srcInd}, even in ideal conditions, the localization results from SRP-PHAT contain many peaks of varying heights. This is due to summing the cross-correlation responses of a non-linear array (Eq. \ref{eq:srpSumInd}). If the array was linear, the localization circles from each pair would all overlap completely. In case of a tetrahedral array, the localization circles from the possible microphone pairs are not co-planar. This is because all the edges of a tetrahedron point in the different directions. The obvious problem here is in multi-source detection. If multiple sources are playing at different levels, how do we determine if a detected peak is a real source or a pseudo-source from another higher level source? The methods to do so form the basis of deconvolution methods for SRP-PHAT.

\subsection{Deconvolution history}
Coming Soon!
\newpage
\subsubsection{Product-SRP-PHAT}
A simple deconvolution approach could be to penalize sources which are only detected by a subset of the microphone pair combinations. This could be done by taking a product and not a sum in Eq. \ref{eq:srpSumInd}.
\begin{equation}
    S_{SRP}(\theta,\phi)=\prod\limits_{i=1}^{M-1}{R_{x_0,x_i}[f_{0,i}(\theta,\phi)]}
     \label{eq:srpProdInd}
\end{equation}
This way, if a peak is caused by a single localization circle, the cross-correlation values from other microphone pairs would be close to zero, and thus would scale the false peak down. The localization results from this are given in fig. \ref{fig:4mic1srcNoisyProd}.
\begin{figure}[H]
    \centering
    \begin{subfigure}[b]{0.96\textwidth}
    \centering
    \includegraphics[width=0.8\textwidth]{Figures/Ind4mic1srcProd20.png}
\end{subfigure}
\vskip \baselineskip
\begin{subfigure}[b]{0.96\textwidth}
    \centering
    \includegraphics[width=0.8\textwidth]{Figures/Ind4mic1srcProd0.png}
\end{subfigure}
\vskip \baselineskip
\begin{subfigure}[b]{0.96\textwidth}
    \centering
    \includegraphics[width=0.8\textwidth]{Figures/Ind4mic1srcProdNeg6.png}
\end{subfigure}
\caption{Figures depict from top-to-bottom product-SRP-PHAT localization results  with SNR = 20dB, SNR = 0dB, SNR = -6dB}
\label{fig:4mic1srcNoisyProd}
\end{figure}
\newpage
The drawback of using product-SRP-PHAT is that the sound level difference between the different sound sources is lost. In normal SRP-PHAT, the array magnitude response at a particular azimuth and elevation is averaged over all microphone pair combinations. Then the level difference between 2 sources is maintained. In product-SRP-PHAT this would not be the case. However if it is assumed that a particular source will have similar magnitude response for all microphone pairs (which is not a strong assumption in far-field), then taking source power $P_{SRP}={S_{SRP}}^{1/M}$, the level difference can be maintained.  Fig. \ref{fig:4mic2srcNoisyCompare} depicts the results of product-SRP-PHAT after level correction.
\begin{figure}[H]
    \centering
    \begin{subfigure}[b]{0.96\textwidth}
    \centering
    \includegraphics[width=0.8\textwidth]{Figures/Ind4mic1srcProd20Corr.png}
\end{subfigure}
\vskip \baselineskip
\begin{subfigure}[b]{0.96\textwidth}
    \centering
    \includegraphics[width=0.8\textwidth]{Figures/Ind4mic1srcProd0Corr.png}
\end{subfigure}
\vskip \baselineskip
\begin{subfigure}[b]{0.96\textwidth}
    \centering
    \includegraphics[width=0.8\textwidth]{Figures//Ind4mic1srcProdNeg6Corr.png}
\end{subfigure}
\caption{Figures depict from top-to-bottom level corrected product-SRP-PHAT localization results with SNR = 20dB, SNR = 0dB, SNR = -6dB}
\label{fig:4mic2srcNoisyCompare}
\end{figure}
\subsubsection{Minimum power SRP-PHAT}
Another deconvolution approach can be to use the far-field assumption again, to assume that the power received from a single source to all microphone pairs is the same. In that case, if, the minimum power between the microphone pair is assumed to be the true power (instead of summing), peaks which are detected only by a subset of microphone arrays would disappear automatically and the deconvolution problem can be solved directly. Fig. \ref{fig:4mic1srcNoisyMinPow} shows the results. Even in adverse conditions of -6 dB, the algorithm is fairly able to detect the source at $(50\degree, 60\degree)$. 
\begin{figure}[H]
    \centering
    \begin{subfigure}[b]{0.96\textwidth}
    \centering
    \includegraphics[width=0.8\textwidth]{Figures/Ind4mic1srcMin20.png}
\end{subfigure}
\vskip \baselineskip
\begin{subfigure}[b]{0.96\textwidth}
    \centering
    \includegraphics[width=0.8\textwidth]{Figures/Ind4mic1srcMin0.png}
\end{subfigure}
\vskip \baselineskip
\begin{subfigure}[b]{0.96\textwidth}
    \centering
    \includegraphics[width=0.8\textwidth]{Figures/Ind4mic1srcMinNeg6.png}
\end{subfigure}
\caption{Figures depict from top-to-bottom minimum power SRP-PHAT localization results with SNR = 20dB, SNR = 0dB, SNR = -6dB}
\label{fig:4mic1srcNoisyMinPow}
\end{figure}
The drawback of both product-SRP-PHAT and Minimum power SRP-PHAT is that in case of localizing point sources, even a minor error in temperature or wind recordings has the potential of not detecting the sound source completely. In real life however, the source is rarely a point source.

\subsection{Other Deconvolution methods}
CLEAN is an algorithm proposed by Jan Högbom in 1974 \cite{1974A&AS...15..417H} to perform the deconvolution of images in radio astronomy. Högbom observed the sky images being polluted by what he described as the "dirty beam". This "dirty beam" is the distortion introduced in the system output when the input is subject to a point source. It relates to the point spread function (PSF) in optic or the array response in our case. The CLEAN algorithm removes the side lobes of the beamformer when the PSF is known in advance throught simulation or measurement. CLEAN-SC is an extension of the CLEAN algorithm but does not rely on knowing the PSF in advance, it uses the coherence between the sidelobes and the main beam to identify the PSF and retrieve the level of the sources. Usually efficient deconvolution is performed in the frequency domain when the PSF is shift-invariant but in our case the PSF change for every scanned position as the array response is different therefore it can become challenging to perform the deconvolution on a real time system. The CLEAN and CLEAN-SC algorithm and some notation will be introduced as well as the algorithm issues. Methods to overcome those issues are also discussed.

\subsubsection{CLEAN deconvolution of the beamformed maps}

The energy map obtained with the SRP-PHAT algorithm is composed of N sound sources of complex amplitude $\hat{A_{N}}$ convoluted with the array response $p(t)$ and some noise n(t). This "image" I(t) can be expressed as in equation \ref{eq:imageclean}
\begin{equation}
    I(t)=\sum\limits_{n=1}^{N}{A_{n}p(t-t_{n})+n(t)}
    \label{eq:imageclean}
\end{equation}

In real situation it might be difficult to retrieve the sources information in this noisy map. A naive and straightforward solution consist in subtracting the array response of the strongest peak and so on until there is no prominent peak left on the map. By subtracting the array response (PSF), a residual image $I_{m}$ is created.

\begin{equation}
    I_{m}(t)=I(t)-\hat{A}_{m}p(t-\hat{t}_{m})
\end{equation}

This idea is among the line of the CLEAN algorithm where usually a fraction of the PSF is deconvoluted from the map. However this method has several drawbacks such as assuming the main peak to be a source and not an addition of interferences created by other sources. Also it assumes the number of sources to be known in advance. To overcome this problem and retrieve the real sources among interferences, a PSF correlation algorithm has been described in \cite{freedman1995techniques}. The PSF Correlation algorithm define the minimal target mass criterion as follow:

\subsubsection{The PSF correlation algorithm}

    \begin{equation}
        M_{m}=\int_{-\infty}^{\infty}|{(I_{m}(t))}|^2
    \end{equation}
    
    \begin{equation}
        \begin{split}
        M_{m} & =\int_{-\infty}^{\infty}\{{(I(t)-\hat{A}_{m}p(t-\hat{t}_{m})}\}\{{(I(t)-\hat{A}_{m}p(t-\hat{t}_{m})}\}^*dt \\
              & = M + |{\hat{A}_{m}}|^2M_{p}-2Re[{\hat{A}_{m}}{R^*_{pl}(\hat{t}_{m})}]\\
              & = M + |{\hat{A}_{m}}|^2M_{p}-2|{\hat{A}_{m}}||{R_{pl}(\hat{t}_{m})}|cos(\hat{\alpha}_{m}-\phi(\hat{t}_{m})
        \end{split}
    \end{equation}

$M_{p}$ is the mass of the point spread function and $R_{pl}(\hat{t_{m}})=\int_{-\infty}^{\infty}{p^{*}(t-\hat(t_{m})I(i)dt}$ is the cross-correlation function between the image and the PSF.
The intuition of the PSF correlation algorithm is that there exists an optimal position $\hat{t}$ which minimize this criteria. This optimal target position to cancel can then be inputed in the CLEAN Algorithm itself.\\

\subsubsection{CLEAN-SC deconvolution of the beamformed maps}

The CLEAN method is attractive in simulation as the cones intersect in one point but as explained earlier it will never happen in practice, there will never be clean cone intersections in the beamformed results due to microphone error position, speed of source shift or wind effect. It is therefore challenging to know what PSF to substract from the map since there is no clean peak in the map. CLEAN-SC is an addition to the CLEAN methods but no assumption about the PSF is made and the PSF is retrieved using the coherence information between the main lobes and the side lobes. The algorithm also retrieves the sources amplitudes, it is interesting to note that the computation time of the CLEAN-SC is usually twice the one to compute the beamformed map, as stated earlier no real time implementation is considered but it is nevertheless interesting to note. CLEAN-SC \cite{sijtsma2007clean} has been published in 2007 by Pieter Sijtsma and the algorithm basics will be explained in this section. Let's first define the Cross Spectrum Matrix (CSM) $\hat{C}$ given by
\begin{equation}
\hat{C}=
    \begin{bmatrix} 
      C_{11} & C_{12} & \cdots & C_{1m_{0}}\\
      \vdots &  C_{22} &       &  \vdots\\
      \vdots &         & \ddots &  \vdots\\
      C_{m_{0}1} &     &       &  C_{m_{0}m_{0}}\\
    \end{bmatrix}  
\end{equation}
where $C_{mm'}$ is the cross spectrum between microphone m  and m'. It is computed by taking the FFT of the pressure recording $p^{*}_{mk}(t)$ and $p_{m'k}(t)$ averaged over K data blocks. $w_{s}$ is a weight filter like Hamming window.
\begin{equation}
    C_{mm'}=\frac{2}{Kw_{s}T}\sum\limits_{k=0}^{K}[P^{*}_{mk}(f,T)P_{m'k}(f,T)]
\end{equation}

Usually a trimmed CSM is used in order to only get the information of independent pairs of microphones, ie all the possible (m,n) combinations of microphones forming the subset S, with m and n the microphones indices. The trimmed CSM  ${\xoverline{C}_{mn}}$ is defined in equation \ref{eq:trimmedcsm}

\begin{equation}
     \xoverline{C} = 
      \begin{cases} 
       C_{mn} & \text{for } (m,n) \in S \\
       0      & \text{for } (m,n) \notin  S
      \end{cases}
    \label{eq:trimmedcsm}
\end{equation}

Let's now define the degraded CSM  $D^{i}$ when source components are removed and the CSM induced by peak source steering vector , $G^{i}$. The indice i is due to the process being iterative as sources are removed from the map. The source cross powers $B_{jk}$ is defined by the equation \ref{eq:crosspow} 

\begin{equation}
    B_{jk}=w_{j}^{*}\xoverline{C}w_{k}
    \label{eq:crosspow}
\end{equation}

with w, the weight vector defined from the steering vector as in equation \ref{eq:weightcsm}

\begin{equation}
    w=g/\sum\limits_{m,n\in S}(|{(g_{m})}|^2|{(g_{n})}|^2)^{1/2}
    \label{eq:weightcsm}
\end{equation}

The CLEAN-SC algorithm demands the source cross-powers of any scan point to be determined entirely by $G^{i}$.

\begin{equation}
    w_{j}^{*}{\xoverline{D}}^{i-1}w^{i}_{k}=w_{j}^{*}\xoverline{G}^{i}w^{i}_{k} 
\end{equation}

This equation has several solution but we can assume the $G^{i}$ to due to a single coherent source component $h^{(i)}$. 

\begin{equation}
    G^{i}=P_{max}^{(i-1)}h^{i}h^{*(i)}
\end{equation}



\subsection{Deconvolution methods}

Beamforming techniques for sound localization have been study intensively over the last decades and is considered among the state of the art methods for sound localization. The main drawback of the conventional beamforming are the side lobes of the beamform itself which sometimes lead to smearing of the sound source location. In our case, the far field SRP-PHAT algorithm is applied to a tetrahedral array, therefore cross-correlations values at different delays are summed by the beamformer and peaks can appear in the energy map at DOA that don't correspond to the incident plane wave DOA. Those peaks in the SRP-PHAT energy map can mask real sources or even add up to other peaks from other sources, thus creating a fake source in the map. Deconvolution methods removes those peaks to reveal the correct peak. Those algortihms are based on the idea of the point spread function which is the response of the array to a point source (here a plane wave with a given DOA). The array response to a plane wave with a given DOA exhibit a particular response which can be deconvolved from the final energy map. While this problem has motivated the creation of new beamformers, the problem still lie in the method and the beaming itself. New classes of algorithms has been developped to deconvolve the noise signals from the desired steered signal such as CLEAN \cite{sijtsma2007clean}. Even if CLEAN was applied successfully, it is in 2006 that the deconvolution approach for the mapping of acoustic sources (DAMAS) was proposed \cite{brooks2006deconvolution} and was acknowledged a major breakthrough in array processing. At first research was mainly focused on aeroacoustic for the development of near-field sound localization system but it seams that a new enthusiasm has taken over scientists trying to solve other sound localization problems. While the DAMAS method is mainly designed for near field measurements in the range of the array aperture; a new method was proposed in \cite{zhao2015large} and \cite{zhao2017large} where a small aperture array was used to measure source signal in the far field. A review of the underlying principes of the main deconvolution algorithms is first given in the following, and finally specific method for coherent and incoherent sources localization is discussed.




\subsubsection{CLEAN}

CLEAN is an algorithm used to remove the influence of the point spread function(PSF) on the beamformed result. It is possible to remove the side lobes of the beamformer and get a more accurate localization, apparently it requires many microphones. ClEAN-SC is an extension of the algorithm for the case of coherent sources, it performs the deconvolution in the frequency domain which usually assumes a shift-invariant PSF, a deconvolution techniques in the time domain was developed and is names TIDY (add reference).

\subsubsection{DAMAS}

Conventional beamforming produces an output that is dependent on array geometry, size, source distance, and frequency. DAMAS removes those array-dependent beamforming characteristics from the output to give an explicit and non ambiguous source localization and level. The DAMAS use the steering vector $\hat{e}$ of the array defined in section.. and formulates that the output of a standard beamformer with $m_{0}$ microphones can be written as:

\begin{equation}
    Y=\frac{\hat{e}^{T}\hat{G}\hat{e}}{m_0^2}
    \label{eq:DAMASoutputbeamformer}
\end{equation}
where $\hat{G}$ is the Cross Spectral Matrix (CSM) given by
\begin{equation}
\hat{G}=
    \begin{bmatrix} 
      G_{11} & G_{12} & \cdots & G_{1m_{0}}\\
      \vdots &  G_{22} &       &  \vdots\\
      \vdots &         & \ddots &  \vdots\\
      G_{m_{0}1} &     &       &  G_{m_{0}m_{0}}\\
    \end{bmatrix}  
\end{equation}
$G_{mm'}$ is the cross spectrum between microphone m  and m'. It is computed by taking the FFT of the pressure recording $p^{*}_{mk}(t)$ and $p_{m'k}(t)$ averaged over K data blocks. $w_{s}$ is a weight filter like Hamming window.
\begin{equation}
    G_{mm'}=\frac{2}{Kw_{s}T}\sum\limits_{k=0}^{K}[P^{*}_{mk}(f,T)P_{m'k}(f,T)]
\end{equation}

Pressure at the two microphones $p^{*}_{mk}(t)$ and $p_{m'k}(t)$ can be split to form two terms accounting for the amplitude of the signal and the steering vector, such that the cross spectrum for a single source can be written as the product of the squared pressure with the cross spectrum of the steering vectors K:

\begin{equation}
\hat{G}_{n}= X_{n} 
\begin{bmatrix} 
      e_{1}^{-1}^{*}e_{1}^{-1} & e_{1}^{-1}^{*}e_{2}^{-1} & \cdots &   e_{1}^{-1}^{*}e_{m_{0}}^{-1}\\
      e_{2}^{-1}^{*}e_{1}^{-1} & e_{2}^{-1}^{*}e_{2}^{-1} &       &  \vdots\\
         &     & \ddots &   \vdots\\
       &     &   &   e_{m_{0}}^{-1}^{*}e_{m_{0}}^{-1}\\
    \end{bmatrix}_{ n}= X_{n}.K 
\end{equation}   

DAMAS assumes that the total CSM is the sum of all of N sources CSM
\begin{equation}
    \hat{G}=\sum_{n}\hat{G}_{n}
\end{equation}  

Therefore, the output of the output of the beamformer in equation \ref{eq:DAMASoutputbeamformer} can be rewritten as

\begin{equation}
    Y(\hat{e})=\frac{\hat{e}^{T}\sum_{n}\hat{G}_{n}\hat{e}}{m_0^2}=\frac{\hat{e}^{T}\sum_{n}X_{n}K\hat{e}}{m_0^2}=\sum_{n}\frac{\hat{e}^{T}K\hat{e}}{m_0^2} X_{n}
\end{equation}

The left term of the product being rewrote as the product of the propagation matrix A with $X_{n}$.
\begin{equation}
    A=\sum_{n}\frac{\hat{e}^{T}K\hat{e}}{m_0^2}
\end{equation}
Giving the following linear equation
\begin{equation}
    Y = A X_{n}
\end{equation}


\subsubsection{DAMAS-C}

As the classical array beamforming, it relies on the following assumption: noise regions under study are distributions of statistically independent sources. Therefore if coherent sources a present it can produce a distorted output. Therefore an extention to DAMAS has been developped to deal with the coherent sources case.


\subsubsection{Multiposition-DAMAS}

A method has been used to scan a large 2D plane using a small aperture array at several position. It is based on DAMAS and gave good resolution once again there is a lot of microphone on the array. The article proposes to solve the DAMAS problem using Covariance Matrix Fitting.






\subsubsection{Discussion}

Deconvolution methods are of great interest as they recover the source level and position at once. It is computationally heavy but it gives improves results over the classial beamforming method. The research current focus is on planar microphone array and nobody has applied deconvolution to tetrehedral arrays also the drawback of such as system is that they seem to relie on a high number of microphones, the low number of microphones case has not been investigated in term of robustness and multi sources.
Might be useful for coherent sources mapping (ground effect), it also solve the SPL. It is a good method designed for a moving microphone array. Drawback are that the bigger area to investigate, the bigger the aperture of the array must be (For one position). If there is several positions used to measure then it's ok.  More on that soon
%\input{3x6_Blinddeconvolution}
%\input{3x4_Eigenvalue}

\subsection{Hybrid SRP-PHAT}

Peterson \cite{peterson2005hybrid} describes a novel approach for sound localization using a two stage approach in order to reduce the computational load. The first stage roughly identifies the sources locations while the second stage is a modified version of the SRP-PHAT algorithm that only performs a grid search around the estimated location from the first stage.  The method is well suited for near-field localization using large aperture array which is not our requirement but the idea can be adapted in the case of far-field sound localization. Section \ref{sec:TDOA} gives an introduction to TDOA based localization and introduces the cone approximation for the far-field. The idea of the hybrid approach is to do a classical GCC-PHAT estimation to get the relative delays between the sensors. The delays estimates are used to derive the cone intersections which give a location estimate which is then input into a SRP-PHAT algorithm where the search region is constrained around the location estimates. A system overview of the algorithm is given in figure \ref{fig:hybridalgo}.

\begin{figure}[H]
    \centering
    \includegraphics[width=1\textwidth]{Figures/hybridalgo.png}
    \caption{Simplified block diagram of the Hybrid algorithm}
    \label{fig:hybridalgo}
\end{figure}

\subsection{Improvements on SRP}

In \cite{salvati2017exploiting} the author proposes a method to improve the computational efficiency and coherence of the grid search using discreet sampling information where the method is called geometrically sampled grid (GSG). \cite{do2007real} uses Stochastic Region Contraction(SRC) to reduce the computational time of the search. \cite{salvati2014incoherent} introduces an incoherent Frequency Fusion based on a normalized arithmetic mean (NAM) which improves the localization performance of SRP, MVDR and MUSIC. Salvati \cite{salvati2015frequency} introduces a SRP weighted MVDR, which combines machine learning power to the noise resilience of the MVDR beamformer, the method is improved in \cite{salvati2016use} by using SVM training. SRP-WMVDR is proved to be much more resilient to noise and better than SRP-PHAT for SNR up to 0. All of those papers used a microphone array composed of more than 8 microphones. Few experimental data are available for the case of 4 microphones and none for the case of a tetrahedral array, whereby a ULA is mostly used for the different test methods.

\subsection{Algorithm complexity}

\input{5x1_Simulation}
\input{6x1_Conclusion}
\section{SRP-PHAT Measurement 1}

\subsection{Purpose}

The purpose of this experiment is to test the SRP-PHAT algorithm in a controlled environment. An experiment is set up in the Anechoic chamber, a tetrahedral microphone array receives sound waves emitted by a point source. The source position is fixed and the microphone array is rotated on the axis in order to change the wave receiving angle.

\subsection{Diagram of the experiment}



\subsection{AAU number List}


\section{Appendix}



\subsection{Other outdoor propagation effects}
\subsubsection{Spreading loss} 
The sound intensity from an omni-directional sound source drops as a function of distance due to wavefront spreading. The intensity I received at distance r from a source with power P, is given by
\begin{equation}
    I = \frac{P}{4\pi r^2}.
\end{equation}
This is due to spherical propagation, where the surface of the sphere has area $4\pi r^2$. In logarithm form this becomes
\begin{equation}
\begin{split}
    10\log(I) &= 10\log\bigg(\frac{P}{4\pi r^2}\bigg) \\
    L_p &= L_w - 20 \log(r) - 11,
\end{split}
\end{equation}
which means a reduction of $20\log2 = 6 dB$, every doubling of r. This equation assumes uniform omni-directional directivity. For directional sources a Directivity Index DI can be added giving
\begin{equation}
    L_p = L_w + DI - 20 \log(r) - 11.
\end{equation}
It is important to remember that such a directivity can be inherent to the source or might be induced due to the location of the source. An omni-directional source placed on a perfectly reflecting plane can only propagate sound into a hemisphere, in which case the DI is 3 dB.  
An infinite line source can be viewed as a linear array of omni-directional point sources. The wavefront spread is cylindrical (surface area $= 2\pi r$),  which gives
\begin{equation}
\begin{split}
    10\log(I) &= 10\log\bigg(\frac{P}{2\pi r}\bigg) \\
    L_p &= L_w - 10 \log(r) - 8,
\end{split}
\end{equation}
The DI is again 3 dB and the reduction is $10\log2 = 3$ dB, every doubling of r. Highway traffic is modelled in a similar manner, assuming 3 dB drop every doubling of distance.

\subsubsection{Diffraction and barriers}
Barriers are sometimes purposefully built to block the direct path from the sound source to the receiver. Sound reaches the receiver either going through the barrier or by diffracting around the top of the barrier. Ground reflections and multi-path-propagation may lead to multiple diffracted wave paths. For a barrier, the ISO 9613-2 \cite{ISO9613} provides the following equation for loss due to barrier insertion
\begin{equation}
    IL = 10\log\bigg[3+\bigg(C_2\frac{\delta_1}{\lambda}\bigg)C_3K_{met}\bigg],
\end{equation}
where $\lambda=wavelength$. The value of $C_2$ determines if ground reflections are taken care of ($C_2 = 20$) or not ($C_2 = 40$), $C_3$ is a factor to take care of double diffraction due to a barrier of finite thickness (or two thin barriers placed some distance apart), $\delta_1$ is the difference in distance between the direct source-to-receiver path and the wave propagation path caused by the barrier, and $K_{met}$ is a correction factor for average downwind meteorological effects. For thin barriers the equation simplifies to 
\begin{equation}
    IL = 10\log\bigg(3+40\frac{\delta_1}{\lambda}\bigg). 
\end{equation}
Over large distances even buildings act like barriers, with the rooftop causing double diffraction. ISO 9613-2 \cite{ISO9613} provides a simple empirical method to calculate attenuation due to buildings.


\printbibliography
\end{document}