\section{Minutes from 02/02/2018}

\begin{itemize}
    \item The purpose of the project is finding major sound contributors in outdoor conditions, for eg. near a highway with a factory in the background. This requires estimating the DOA of multiple sound sources in far-field while characterizing the sound sources in terms of frequency content and sound level.
    \item The signal spectra considered would be at the low to mid frequencies and with multiple sound sources. Moving sources are not to be tracked real-time, in that this project is not about real time tracking, rather trying to create a sound-scape with major sound contributors. It would be interesting to run simulations to find the optimal dimension/ geometry of the array given a particular signal spectra.
    \item Since the array will be outdoors, measurements will be affected by several weather conditions like wind and temperature. The scope of which weather factors to consider for the project needs to be defined.
    \item Since the project is not about real time tracking, we can process signal from a large time window (or multiple small time windows mean square averaged). Example "How much noise was there over the last 30 minutes? What is the direction, level and frequency content of the different sources?"
    \item As the noise sources characteristics varies a lot (ex impulsive sound, siren, low frequency rumbling) a fusion approach can be considered. Multiple algorithms can be tested and used individually or fused together depending on the measurement conditions. The scope of this needs to be defined after running some simulations.
    \item Another approach can be to match TDOA of simulated sources to the real world measurements (create and tune a simulated model).
\end{itemize}

\section{Broad timeline}

\begin{itemize}
    \item \textbf{02-23 Feb} &- Narrow down the scope: Clearly define test cases and simulations that will be run before measurements are to be done. The theoretical writing part of the thesis and project abstract should be close to completion along with this.
    \item \textbf{24 Feb-4 Mar} &- Define the array geometry that will be used to run actual measurements.
    \item \textbf{5-20 Mar} &- Test the different cases in the simulation .
    \item \textbf{21 Mar-21 Apr} &- Run measurements.
    \item \textbf{22 Apr-15 May} &- Compile and analyze results.
    \item \textbf{30 May} &- Finish up the thesis report.
\end{itemize}

\section{2/16 Meeting}
\begin{itemize}
    \item Censor, patent etc
    \item On site visit
    \item Array geometry
    \item 
\end{itemize}
