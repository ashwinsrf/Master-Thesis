\section{Appendix}
\subsection{papers}
\begin{itemize}
    \item DOA estimation of moving sound sources in the context of nonuniform spatial noise using acoustic vector sensor:  \\
    https://link.springer.com/article/10.1007/s11045-013-0273-0 
    \item : A Real-Time 3D Sound Localization System with
    Miniature Microphone Array for Virtual Reality\\
    http://ieeexplore.ieee.org/abstract/document/6361029/?reload=true
\end{itemize}

\subsection{karim mail}
Here is attached a list of papers. Some are very advanced, you could skip them at first. You could read first for instance Brandstein, Japs, Pertila (not necessary all content).

$- brandstein_phd_1995.pdf: old, but good introduction in Chapters 2 and 3
- PhD_Thesis-Daniele_Salvati.pdf: https://www.google.dk/url?sa=t&rct=j&q=&esrc=s&source=web&cd=1&ved=0ahUKEwjK5YSKzOHYAhXFYlAKHSsfCswQFggrMAA&url=https%3A%2F%2Fdspace-uniud.cineca.it%2Fbitstream%2F10990%2F116%2F1%2FPhD_Thesis-Daniele_Salvati.pdf&usg=AOvVaw1A6eSja417vSiC-u_v0zxw

- jasp_2006.pdf: discussion about time-delay estimation, used for TDOA approaches
- pertila.pdf: recent thesis on topic
Multi sources:
- sam2000_1.pdf (using clustering approach).$
- scheuing-ieeetasl.pdf (a correlation-based technique for identifying the time delays of multiple sources)

Tetrahedral array
- 20150911104438393.pdf: recent publication on tetrahedral array

\subsection{interesting links}

\begin{itemize}
    \item Good figure export toolbox: https://se.mathworks.com/matlabcentral/fileexchange/23629-export-fig
\end{itemize}


\subsection{Sound wave types}
Before describing how the sound propagation is affected by different factors, it is important to describe the components that make up an outdoor sound field. They are [THIS SECTION NEEDS MORE RESEARCH!!]
\begin{itemize}
    \item \textbf{Creeping wave}: The component of the sound field that 'creeps' along a surface. The wave near to the surface conducts sound along a 'minimum-time path' to the receiver. If the atmosphere curves the sound waves upwards, the direct propagation of sound might not be possible. The sound from the source then curves along the ground. The same effect happens if the ground itself is sloping downwards from the source towards the receiver. The creeping wave loses energy as it propagates as it continuously radiates sound away from the surface. 
    
    [DIAGRAM FOR CREEPING WAVE]
    
    \item \textbf{Ground wave}: 
    
\end{itemize}