\section{Introduction}



The subject of this thesis is to investigate the robustness of different localization algorithm for the specific case of sound sources in outdoor conditions. The main application of this work is in environmental sound assessment where a microphone array is placed outside and the major sounds sound sources present on the field needs to be assessed. This includes estimating the direction of arrival (DOA) of multiple sound sources in the far-field while characterizing the sound sources in terms of frequency content and sound level. The sound sources encountered outside varies a lot, this thesis focus mainly on sources signal spectrum ranging from low to mid frequencies. Acoustics Sound localization(ASL) has been successfully applied to a wide range of engineering applications. The problems addressed previously includes mostly speech enhancement system, speaker identification and teleconferencing in the speech environment where a robust algorithm under high background noise, reverberant conditions was needed. This thesis does not include moving sources since it is not a tracking problem that we are trying to solve. This thesis solves the problem with the constraint of using as few microphones as possible while being robust and therefore includes a discussion on the different array geometry applicable. The microphone array is outdoor, therefore signals will be affected by several weather conditions like wind and temperature therefore a model for the signal outdoor propagation propagation is also investigated. A literature review of the main sound localization algorithms is also present. Finally an hybrid algorithm is proposed and its performance is tested in simulation and real-life situation.




