\section{Introduction}

Sound has been a subject of fascination and investigation for a long time. We have known since prehistoric times that sound travels slower than light, evermore proven whenever a flash of lightning was seen before the clap was heard \cite{ampel1993history}. Marin Mersènne of Paris used a cannon and a pendulum to make speed of sound measurements in 1636. We have come a long way since then, whereupon it is of interest to deduce the complex sound field behaviour outdoors. This is required to solve a two-part problem. Firstly, ascertaining urban noise levels, and secondly, determining their major contributors. The problem can solved using various source localization algorithms, which can be used to 'localize' sound received at a microphone array to its source.

Sound localization algorithms have been successfully applied (to varying degrees) to a wide range of engineering problems. A significant bit of research has been done on speech enhancement systems, whereby speaker identification and teleconferencing in an environment having high background noise and reverberant conditions was needed. Outdoor sound can be appreciably different from this situation. The subject of this thesis is then to investigate the robustness of different localization algorithms for the specific case of sound sources in outdoor conditions. The main application of this work is in environmental sound assessment, where a microphone array is placed outside and the major sounds sound sources present in the field are assessed. This includes estimating the direction of arrival (DOA) of multiple sound sources in the far-field while characterizing the sound sources in terms of frequency content and sound level. The sound source spectra encountered outside vary a lot. This thesis focus mainly on source signal spectrum ranging from low-to-mid frequencies. It should be noted that the purpose of this thesis is not to track moving sources, rather the thesis tackles the problem of \'outdoor sound levels and their contributors\' with the constraint of using as few microphones as possible while being robust to different noise and weather conditions. This entails an investigation into the different microphone array geometries applicable and their possible advantages/ disadvantages, provided in the thesis. A variety of factors affect outdoor sound propagation like the ground (earth), atmospheric absorption, barriers between line-of-sight to the sound source, wind, temperature, turbulence etc. A short overview of outdoor sound propagation modelling based on these factors is provided. A literature review of the main sound localization algorithms is given. Finally a hybrid algorithm is proposed and its performance is tested in model-based simulations and real-life conditions.




